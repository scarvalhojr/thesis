\documentclass[11pt,a4paper]{letter}
\usepackage{ngerman}

\address{Viktoriastr. 15, D-33602 Bielefeld}
\name{S\'ergio Anibal de Carvalho Junior}
\date{27. Februar 2007}

\begin{document}
\begin{letter}{
  An den Dekan der Technischen Fakult\"at\\
  Universit\"at Bielefeld\\
  Postfach 100131\\
  33501 Bielefeld
}

\opening{ {\bf Vorschlag zur Benennung der Gutachter}}

Ich schlage Herrn \textbf{Dr.\ Sven Rahmann}, Technische Fakult\"at der
Universit\"at Bielefeld, sowie Herrn \textbf{Prof.\ Dr.\ Robert Giegerich},
Technische Fakult\"at der  Universit\"at Bielefeld, als Gutachter vor.

{\bf Vorschlag zur Benennung der Pr\"ufungskommission}

Nach Absprache mit den Beteiligten schlage ich die folgenden vier Personen als
Mitglieder der Pr\"ufungskommission vor:
\begin{itemize}
  \item Prof.\ Dr.\ Ralf \textbf{M\"oller} (Vorsitz),
  \item Prof.\ Dr.\ Jens \textbf{Stoye},
  \item Prof.\ Dr.\ Robert \textbf{Giegerich},
  \item Dr.\ Sven \textbf{Rahmann} (akademischer Mittelbau).
\end{itemize}

Herr M\"oller hat sich bereiterkl\"art, den Vorsitz zu \"ubernehmen.

\closing{Mit freundlichen Gr\"u{\ss}en}

\encl{Publikationsliste}

\newpage
{\bf\large Publikationen}

\begin{itemize}
\item S. A. de Carvalho Jr. and S. Rahmann. Modeling and optimizing
oligonucleotide microarray layout. In I. Mandoiu and A. Zelikowsky, editors,
\emph{Bioinformatics Algorithms: Techniques and Applications}, Wiley Book Series
on Bioinformatics. Wiley, 2007. To appear.

\item S. A. de Carvalho Jr. and S. Rahmann. Improving the layout of
oligonucleotide microarrays: Pivot Partitioning. In P. Bucher and B. Moret,
editors, \emph{Proceedings of the 6th Workshop of Algorithms in Bioinformatics},
volume 4175 of \emph{Lecture Notes in Computer Science (LNCS)}, pages 321--332.
Springer, 2006.

\item S. A. de Carvalho Jr. and S. Rahmann. Microarray layout as a quadratic
assignment problem. In D. Huson, O. Kohlbacher, A. Lupas, K. Nieselt, and A.
Zell, editors, \emph{Proceedings of the German Conference on Bioinformatics},
volume P-83 of \emph{Lecture Notes in Informatics (LNI)}, pages 11--20.
Gesellschaft f\"ur Informatik, 2006.

\item S. A. de Carvalho Jr. and S. Rahmann. Searching for the shortest common
supersequence. Technical Report 2005-03, Technische Fakult\"at der Universit\"at
Bielefeld, Apr 2005.
\end{itemize}

\end{letter}
\end{document}
