\documentclass[a4paper,11pt]{letter}
\usepackage{ngerman}

\address{Viktoriastr. 15, D-33602 Bielefeld}
\name{S\'ergio Anibal de Carvalho Junior}

\begin{document}
\begin{letter}{
  An den Dekan der Technischen Fakult\"at\\
  Universit\"at Bielefeld\\
  Postfach 100131\\
  33501 Bielefeld
}

\opening{ {\bf Zulassung Promotionsverfahren}}

Sehr geehrter Prof. Dr. Ritter,

hiermit beantrage ich die Zulassung zum Promotionsverfahren an der Technischen
Fakult\"at der Universit\"at Bielefeld nach \S3 der Promotionsordnung. Ich
erkl\"are, dass ich die Promotionsordnung der Fakult\"at in der Fassung vom 15.\ 
M\"arz 2002 zur Kenntnis genommen habe. Es wurde bisher kein Promotionsverfahren
f\"ur mich er\"offnet.

\closing{Mit freundlichen Gr\"u{\ss}en}

\encl{Publikationsliste,\\
  Kopie des Diplomzeugnisses\\
  Lebenslauf und Darstellung des Bildungsganges
}

\newpage
{\bf\large Lebenslauf}

{\bf Personal Information}\\
Name: S\'ergio Anibal de Carvalho Junior\\
Birth: 1 July 1975, Salvador, Brazil\\
Marital status: married\\
Citizenship: Brazilian\\
Languages: Portuguese (native), English (fluent), German (conversational)\\

{\bf Education}
\begin{itemize}

\item M.Sc. in Advanced Computing, with Distinction\\
      King's College London, London, United Kingdom\\
      September 2003

\item Advanced Postgraduate Specialization in Distributed Systems\\
      Laborat\'orio de Sistemas Distribu\'idos (LaSid), Universidade Federal da
      Bahia, Salvador, Brazil\\
      March 2002

\item B.Sc. Computing Science\\
      Universidade Federal da Bahia, Salvador, Brazil\\
      September 2000

\end{itemize}

{\bf Awards}
\begin{itemize}

\item Chevening Scholarship\\
      British Council, Foreign \& Commonwealth Office, United Kingdom, included
      King's College London's tuition fees and living expenses\\
      September 2002 -- August 2003

\end{itemize}

{\bf Employment}
\begin{itemize}

\item Systems Support Coordinator\\
      Telecom Italia Mobile, mobile phone service provider, Salvador, Brazil\\
      October 2000 -- August 2002

\item Systems Analyst and Software Developer\\
      RMO Consultores Associados, software development, Salvador, Brazil\\
      May 1997 -- October 2000

\end{itemize}

\newpage
{\bf\large Publikationen}

\begin{itemize}
\item S. A. de Carvalho Jr. and S. Rahmann. Modeling and optimizing
oligonucleotide microarray layout. In I. Mandoiu and A. Zelikowsky, editors,
\emph{Bioinformatics Algorithms: Techniques and Applications}, Wiley Book Series
on Bioinformatics. Wiley, 2007. To appear.

\item S. A. de Carvalho Jr. and S. Rahmann. Improving the layout of
oligonucleotide microarrays: Pivot Partitioning. In P. Bucher and B. Moret,
editors, \emph{Proceedings of the 6th Workshop of Algorithms in Bioinformatics},
volume 4175 of \emph{Lecture Notes in Computer Science (LNCS)}, pages 321--332.
Springer, 2006.

\item S. A. de Carvalho Jr. and S. Rahmann. Microarray layout as a quadratic
assignment problem. In D. Huson, O. Kohlbacher, A. Lupas, K. Nieselt, and A.
Zell, editors, \emph{Proceedings of the German Conference on Bioinformatics},
volume P-83 of \emph{Lecture Notes in Informatics (LNI)}, pages 11--20.
Gesellschaft f\"ur Informatik, 2006.

\item S. A. de Carvalho Jr. and S. Rahmann. Searching for the shortest common
supersequence. Technical Report 2005-03, Technische Fakult\"at der Universit\"at
Bielefeld, Apr 2005.
\end{itemize}

\end{letter}
\end{document}
