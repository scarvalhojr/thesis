%%%%%%%%%%%%%%%%%%%%%%%%%%%%%%%%%%%%%%%%%%%%%%%%%%%%%%%%%%%%%%%%%%%%%%%%%%%%%%%%
\chapter{MLP and the Quadratic Assignment Problem}
\label{ch:qap}
%%%%%%%%%%%%%%%%%%%%%%%%%%%%%%%%%%%%%%%%%%%%%%%%%%%%%%%%%%%%%%%%%%%%%%%%%%%%%%%%

In this chapter, we show that the Microarray Placement Problem (MLP) with
general distance-dependent and position-dependent weights is an instance of the
\emph{Quadratic Assignment Problem} (QAP), a classical combinatorial
optimization problem introduced by~\citet{Koopmans1957}, which opens up the way
for using QAP techniques to design microarray chips.

We then use an existing QAP heuristic algorithm called GRASP to design the
layout of small artificial chips, comparing our results with the best known
placement algorithm. The chapter ends with a discussion about how this approach
can be combined with other existing algorithms to design and improve larger
microarrays.

%%%%%%%%%%%%%%%%%%%%%%%%%%%%%%%%%%%%%%%%%%%%%%%%%%%%%%%%%%%%%%%%%%%%%%%%%%%%%%%%
\section{Quadratic assignment problem}
\label{sec:qap_qap}

The quadratic assignment problem (QAP) can be stated as follows. Given
$n \times n$ real-valued matrices $F = (f_{ij})\geq 0$ and $D = (d_{kl})\geq 0$,
find a permutation $\pi$ of $\{1, 2, \ldots n\}$ such that
\begin{equation}\label{eq:qap_def}
  \sum_{i=1}^{n} \sum_{j=1}^{n}\,  f_{ij} \cdot d_{\pi(i)\pi(j)} \to \min.
\end{equation}

The attribute \emph{quadratic} stems from the fact that the target function can
be written with $n^2$ binary indicator variables $x_{ik}\in\{0,1\}$, where
$x_{ik}:=1$ if and only if $k=\pi(i)$. The objective~(\ref{eq:qap_def}) then
becomes
%%
\[
  \sum_{i=1}^{n} \sum_{j=1}^{n}\,  f_{ij} \cdot 
  \sum_{k=1}^{n} \sum_{l=1}^{n}\,  d_{kl} \cdot x_{ik}\cdot x_{jl}
  \to \min,
\]
%%
such that $\sum_{k}\, x_{ik}=1$ for all $i$, $\sum_{i}\, x_{ik}=1$ for all $k$,
and $x_{ik}\in\{0,1\}$ for all $(i,k)$. The objective function is a quadratic
form in $x$.

The QAP has been used to model a variety of real-life problems. One common
example is the facility location problem where $n$ facilities must be assigned
to $n$ locations. The facilities could be, for instance, the clinics, doctors or
services (X-ray, emergency room, etc.) provided by a hospital and the locations
could be the available rooms of the hospital building.

In this scenario, $F$ is called the \emph{flow matrix} as $f_{ij}$ represents
the flow of materials or persons from facility $i$ to facility $j$. Matrix $D$
is called the \emph{distance matrix}, as $d_{kl}$ gives the distance between
locations $k$ and $l$. One unit of flow is assumed to have an associated cost
proportional to the distance between the facilities, and the optimal permutation
$\pi$ defines a one-to-one assignment of facilities to locations with minimum
cost.

%%%%%%%%%%%%%%%%%%%%%%%%%%%%%%%%%%%%%%%%%%%%%%%%%%%%%%%%%%%%%%%%%%%%%%%%%%%%%%%%
\section{QAP formulation of the MLP}
\label{sec:qap_mlp}

The MLP can be seen as an instance of the QAP, where we want to find a
one-to-one correspondence between spots and probes minimizing a given penalty
funtion such as total border length or total conflict index (defined in Chapter
\ref{ch:mlp}). To formulate it, we use the facility location example by viewing
the probes as locations and the spots as facilities, i.e., the spots are
assigned to the probes. The flow matrix $F$ then contains the ``closeness''
values between spots, while the distance matrix $D$ contains the conflicts
between probe embeddings.

We first give the general formulation for conflict index minimization case; the
border length minimization case is obtained by using the particular weight
functions given in Section~\ref{sec:mlp_bl_vs_ci}.

In a realistic setting, we may have more spots available than probes to place.
Below, we show that this does not cause problems as we can add enough ``empty''
probes and define their weights appropriately.

Perhaps more severely, we assume that all probes have a single pre-defined
embedding in order to force a one-to-one relationship.  A more elaborate
formulation would consider all possible embeddings of a probe, but then it
becomes necessary to ensure that only one embedding of each probe is used. This
still leads to a quadratic integer programming problem, albeit with slightly
different side conditions.

Our goal is to design a microarray minimizing the sum of conflict indices over
all spots~$s \in \CalS$, i.e.,
%%
\[
\sum_{s \in \CalS} \mathcal{C}(s) \to \min.
\]

The ``flow'' $f_{ij}$ between spots $i$ and $j$ depends on their distance on the
chip; in accordance with the conflict index model, we set
%%
\begin{equation}
  f_{ij} := \Ind{i,j \text{ neighbors}} \cdot \gamma(i,j)
\end{equation}
%%
where ``neighbors'' means that spots $i$ and $j$ are at most three cells away
(horizontally and vertically) from each other. Note that most of the flow values
on large arrays are zero. For border length minimization, the case is even
simpler: We set $f_{ij}:=1$ if spots $i$ and $j$ are adjacent, and $f_{ij}:=0$
otherwise.

The ``distance'' $d_{kl}$ between probes $k$ and $l$ depends on the conflicts
between their embeddings $\eps_k$ and $\eps_l$. To account for possible
``empty'' probes to fill up surplus spots, we set $d_{kl}:=0$ if $k$ or $l$ or
both refer to an empty probe --- i.e., empty probes never contribute to the
target function since we do not mind if nucleotides are erroneously synthesized
on spots assigned to empty probes. For real probes, we set
%%
\begin{equation}
  d_{kl} := \sum_{t=1}^{T} \Bigl(
    \Ind{\eps_{k,t}=0}
    \cdot \omega(\eps_k,t)
    \cdot \Ind{\eps_{l,t}=1} \Bigr).
\end{equation}

Note that $d_{kl}$ is related to the conflict index distance $C(k,l)$ defined in
Section~\ref{sec:mlp_conflict_index} (Equation~\ref{eq:ci_dist}):
%%
\begin{eqnarray*}
 &   & d_{kl} + d_{lk} \\
 & = & \sum_{t=1}^{T} \Bigl( \Ind{\eps_{k,t}=0} \cdot \omega(\eps_k,t) \cdot \Ind{\eps_{l,t}=1} \Bigr)
     + \sum_{t=1}^{T} \Bigl( \Ind{\eps_{l,t}=0} \cdot \omega(\eps_l,t) \cdot \Ind{\eps_{k,t}=1} \Bigr) \\
 & = & \sum_{t=1}^{T} \Bigl( \Ind{\eps_{k,t}=0 \text{ and } \eps_{l,t}=1} \cdot \omega(\eps_k,t) \Bigr)
     + \sum_{t=1}^{T} \Bigl( \Ind{\eps_{l,t}=0 \text{ and } \eps_{k,t}=1} \cdot \omega(\eps_l,t) \Bigr) \\
 & = & \sum_{t=1}^{T} \Bigl(
                        \Ind{\eps_{k,t}=0 \text{ and } \eps_{l,t}=1} \cdot \omega(\eps_k,t) +
                        \Ind{\eps_{l,t}=0 \text{ and } \eps_{k,t}=1} \cdot \omega(\eps_l,t)
                      \Bigr) \\
 & = & C(k,l)
\end{eqnarray*}

In the case of border length minimization, where $\theta=0$ and $c=1/2$ (see
Section~\ref{sec:mlp_bl_vs_ci}), we obtain that
$d_{kl} + d_{lk} = H(k,l) = H(l,k)$, where $H_{kl}$ denotes the Hamming distance
between the embeddings $\eps_k$ and $\eps_l$ (Equation~\ref{eq:hamming}).

It now follows that for a given assignment $\pi$, we have,
%%
\[
f_{ij} \cdot d_{\pi(i)\pi(j)} = \sum_{t=1}^{T} \Bigl(
  \Ind{\eps_{\pi(i),t}=0}
  \cdot \omega(\eps_{\pi(i)},t)
  \cdot \Ind{\eps_{\pi(j),t}=1}
  \cdot \Ind{i,j \text{ neighbors}}
  \cdot \gamma(i,j) \Bigr).
\]

The objective function~(\ref{eq:qap_def}) then becomes
%%
\begin{eqnarray*}
 &   & \sum_i \sum_j\, f_{ij} \cdot d_{\pi(i)\pi(j)} \\
 & = & \sum_i \sum_j\, \sum_{t=1}^{T}
                       \Bigl(
                         \Ind{\eps_{\pi(i),t}=0}
                         \cdot \omega(\eps_{\pi(i)},t)
                         \cdot \Ind{\eps_{\pi(j),t}=1}
                         \cdot \Ind{i,j \text{ neighbors}}
                         \cdot \gamma(i,j)
                       \Bigr) \\
 & = & \sum_i \sum_{t=1}^T\,
              \Bigl(
                 \Ind{\eps_{\pi(i),t}=0}
                 \cdot \omega(\eps_{\pi(i)},t)
                 \cdot \sum_j\,
                 \Ind{i,j \text{ neighbors}}
                 \cdot \Ind{\eps_{\pi(j),t}=1}
                 \cdot \gamma(i,j)
              \Bigr) \\
 & = & \sum_i \sum_{t=1}^T\,
              \Bigl(
                 \Ind{\eps_{\pi(i),t}=0}
                 \cdot \omega(\eps_{\pi(i)},t)
                 \cdot \sum_{\substack{j \text{: neighbor}\\\text{of } i}}\,
                 \Ind{\eps_{\pi(j),t}=1}
                 \cdot \gamma(i,j)
              \Bigr) \\
 & = & \sum_i \mathcal{C}(i), \\
\end{eqnarray*}
%%
and indeed equals the total conflict index with our definitions of $F=(f_{ij})$
and $D=(d_{kl})$.

\paragraph{Remark.}
Note that it is technically possible to switch the definitions of $F$ and $D$,
i.e., to assign probes to spots instead of spots to probes as we do now, without
modifying the mathematical problem formulation. However, this would lead to high
distance values for neighboring spots and many zero distance values for
independent spots, a somewhat counterintuitive model. Also, some QAP heuristics
initially find pairs of objects with large flow values and place them close to
each other. Therefore, the way of modeling $F$ and $D$ may be significant.

%%%%%%%%%%%%%%%%%%%%%%%%%%%%%%%%%%%%%%%%%%%%%%%%%%%%%%%%%%%%%%%%%%%%%%%%%%%%%%%%
\section{QAP heuristics}
\label{sec:qap_heuristics}

We have shown how the microarray placement problem can be modeled as a
quadratic assignment problem. However, the QAP is known to be NP-hard and
particularly hard to solve in practice. Instances of size larger than
$n = 20$ are generally considered to be impossible to solve to
optimality. Fortunately, several heuristics exist, including approaches based on
tabu search, simulated annealing and genetic algorithms
\citep[for a survey, see][]{Cela1997}. Our formulation is thus of interest
because we can now use existing QAP heuristics to design the layout of
microarrays minimizing either the sum of border lengths or conflict indices.

As an example, we briefly describe a general QAP heuristic known as GRASP \citep
{Li1994}, which was first used for solving the QAP by \citet{Feo1995}, and an
improved version called GRASP with path-relinking \citep{Oliveira2004}, that we
used to design small microarray chips with our formulation.

\subsection{GRASP with Path-relinking}
\label{sec:qap_grasp}

GRASP (Greedy Randomized Adaptive Search Procedure) is comprised of two phases:
a construction phase where a random feasible solution is built, and a local
search phase where a local optimum in the neighborhood of that solution is
sought. In the following description we use the terms of the facility location
problem: $f_{ij}$ is the flow between facilities $i$ and $j$, $d_{kl}$ is the
distance between locations $k$ and $l$.

The construction phase starts by sorting the $(n^2 - n)$ elements of the
distance matrix in increasing order and keeping the smallest
$E:= \lfloor \beta (n^2 - n) \rfloor$ elements, where $0 < \beta < 1$ is a
restriction parameter given as input.
%%
\begin{displaymath}
d_{k_1 l_1} \le d_{k_2 l_2} \le \cdots \le d_{k_E l_E}.
\end{displaymath}

Similarly, the $(n^2 - n)$ elements of the flow matrix are sorted, this time in
decreasing order, and the largest $E$ elements are kept:
%%
\begin{displaymath}
f_{i_1 j_1} \ge f_{i_2 j_2} \ge \cdots \ge f_{i_E j_E}.
\end{displaymath}

Then, the costs of assigning pairs of facilities to pairs of locations are
computed. The cost of initially assigning facility $i_q$ to location $k_q$ and
facility $j_q$ to location $l_q$ for some $q\in\{1,\ldots,E\}$ is
$d_{k_q l_q} f_{i_q j_q}$. GRASP sorts the vector
%%
\begin{displaymath}
(d_{k_1 l_1} f_{i_1 j_1},\;
 d_{k_2 l_2} f_{i_2 j_2},\; \ldots,\;
 d_{k_E l_E} f_{i_E j_E}),
\end{displaymath}
%%
keeping the $\lfloor \alpha E \rfloor$ smallest elements, where $0 < \alpha < 1$
is another restriction parameter. A simultaneous assignment of a pair of
facilities to a pair of locations is selected at random among those with the
$\lfloor \alpha E \rfloor$ smallest costs, and a feasible solution is then built
by making a series of greedy assignments.

In the local search phase, GRASP searches for a local optimum in the
neighborhood of the constructed solution. Several search strategies and
definitions of neighborhood can be used. One possible approach is to check every
possible swap of assignments and make only those which improve the current
solution until no further improvements can be made.

The construction and local search phases are repeated for a given number of
times, and the best solution found is returned.

\paragraph{Path-relinking.}
GRASP takes no advantage of the knowledge gained in previous iterations to
build or improve an obtained solution, i.e., each new solution is built from
scratch.

GRASP with path-relinking is an extension of the basic GRASP algorithm that uses
an ``elite set'' to store the best solutions found. It incorporates a third
phase that chooses, at random, one elite solution that is used to improve the
solution produced at the end of the local search phase.

Solutions $p$ and $q$ are combined as follows. For every location
$k = 1, \ldots, n$, the path-relinking algorithm attempts to exchange facility
$p_k$ assigned to location $k$ in  solution $p$ with facility $q_k$ assigned to
location $k$ in the elite solution. In order to keep the solution $p$ feasible,
it exchanges $p_k$ with $p_l$, where $p_l = q_k$. This exchange is performed
only if it results in a better solution. The result of the path-relinking phase
is a solution $r$ that is at least as good as the better of $p$ and $q$.

%%%%%%%%%%%%%%%%%%%%%%%%%%%%%%%%%%%%%%%%%%%%%%%%%%%%%%%%%%%%%%%%%%%%%%%%%%%%%%%%
\section{Results}
\label{sec:qap_results}

We present experimental results of using GRASP with path-relinking (GRASP-PR)
for designing the layout of small artificial chips, and compare them with the
layouts produced by the Greedy placement algorithm (described in
Section~\ref{sec:placement_greedy}), with the number $Q$ of candidates per spot
set to a sufficiently large value so that all available probes are considered
for each spot.

We used a C implementation of GRASP-PR provided by \citet{Oliveira2004} with
default parameters (32 iterations, $\alpha=0.1$, $\beta=0.5$, and elite set of
size $10$). The main routine takes three arguments: the dimension $n$ of the
problem (in our case, the number of spots or probes) and matrices $F$ and $D$.
The matrices were generated using the formulations presented in Section
\ref{sec:qap_mlp}.

The data set consists of chips with probes of length 25 uniformly generated and
asynchronously embedded in a deposition sequence of length 74. The running times
and the border lengths of the resulting layouts are shown in
Table~\ref{tab:graspr_greedy_bl} (all results are averages over a set of ten
chips).

\begin{table}[t]\centering
\caption{\label{tab:graspr_greedy_bl}
  Total border length of random chips compared with the layouts produced by
  Greedy and GRASP with path-relinking. Reductions in border length are reported
  in percentages compared to the random layout.}
\small{
\begin{tabular*}{\hsize}{crcrrrcrrr}
          & Random & & \multicolumn{3}{c}{Greedy placement}  & & \multicolumn{3}{c}{GRASP with path-relinking}  \\ \cline{2-2} \cline{4-6} \cline{8-10}
Chip      & Border & & Border & Reduction & Time             & & Border & Reduction & Time   \\
dimension & length & & length & (\%)      & (sec.)           & & length & (\%)      & (sec.) \\
\hline
$6\times 6$   & 1\,989.20 & & 1\,714.60 & 13.80 & 0.01 & & 1\,672.20 & 15.94 &   2.73 \\
$7\times 7$   & 2\,783.20 & & 2\,354.60 & 15.40 & 0.02 & & 2\,332.60 & 16.19 &   6.43 \\
$8\times 8$   & 3\,721.20 & & 3\,123.80 & 16.05 & 0.03 & & 3\,099.13 & 16.72 &  12.49 \\
$9\times 9$   & 4\,762.00 & & 3\,974.80 & 16.53 & 0.05 & & 3\,967.20 & 16.69 &  25.96 \\
$10\times 10$ & 5\,985.20 & & 4\,895.60 & 18.20 & 0.06 & & 4\,911.40 & 17.94 &  47.57 \\
$11\times 11$ & 7\,288.40 & & 5\,954.40 & 18.30 & 0.10 & & 5\,990.73 & 17.80 &  87.48 \\
$12\times 12$ & 8\,714.00 & & 7\,086.20 & 18.68 & 0.11 & & 7\,159.80 & 17.84 & 152.42 \\
\hline
\end{tabular*}}
\end{table}

Our results show that GRASP-PR produces layouts with lower border lengths than
Greedy on the smaller chips. On $6\times 6$ chips, GRASP-PR outperforms Greedy
by $2.14$ percentage points on average ($15.94\% - 13.80\%$), when compared to
the initial random layout. On $9\times 9$ chips, however, this difference drops
to $0.16$ percentage points, while Greedy generates better layouts on
$11\times 11$ or larger chips. In terms of running time, Greedy is faster and
shows little variation as the number of probes grows. In contrast, the time
required to compute a layout with GRASP-PR increases at a fast rate.

Table~\ref{tab:graspr_greedy_ci} shows better results in terms of conflict
indices. GRASP-PR consistently produces better layouts on all chip dimensions,
achieving up to $6.38$\% less conflicts on $10\times 10$ chips, for example,
when compared to Greedy. In terms of running times, GRASP-PR is even slower than
in the border length case. The reason is not clear, but it could be because the
distance matrix contains fewer zero entries with the conflict index formulation.

The gains in terms of conflict index of both Greedy and GRASP-PR are clearly
less than the gains in terms of border length (when compared to the initial
random layout). This may be because the probe embeddings are fixed and the
reduction of conflicts is restricted to the relocation of the probes, which only
accounts for one part of the conflict index model.

\begin{table}[t]
\caption{\label{tab:graspr_greedy_ci}
  Average conflict indices of random chips compared with the layouts produced by
  Greedy and GRASP with path-relinking.}
\small{
\begin{tabular*}{\hsize}{crcrrrcrrr}
          & Random & & \multicolumn{3}{c}{Greedy placement}  & & \multicolumn{3}{c}{GRASP with path-relinking}  \\ \cline{2-2} \cline{4-6} \cline{8-10}
Chip      & Avg. C.& & Avg. C.& Reduction & Time             & & Avg. C.& Reduction & Time   \\
dimension & Index  & & Index  & (\%)      & (sec.)           & & Index  & (\%)      & (sec.) \\
\hline
$6\times 6$   & 524.28 & & 495.15 & 5.56 & 0.05 & & 467.08 & 10.91 &   3.68 \\
$7\times 7$   & 558.25 & & 521.90 & 6.51 & 0.07 & & 489.32 & 12.35 &   8.84 \\
$8\times 8$   & 590.51 & & 551.84 & 6.55 & 0.09 & & 515.69 & 12.67 &  19.48 \\
$9\times 9$   & 613.25 & & 568.62 & 7.28 & 0.11 & & 533.79 & 12.96 &  38.83 \\
$10\times 10$ & 628.50 & & 576.49 & 8.28 & 0.11 & & 539.69 & 14.13 &  73.09 \\
$11\times 11$ & 642.72 & & 588.91 & 8.37 & 0.12 & & 551.41 & 14.21 & 145.67 \\
$12\times 12$ & 656.86 & & 598.21 & 8.93 & 0.12 & & 561.21 & 14.56 & 249.19 \\
\hline
\end{tabular*}}
\end{table}

%%%%%%%%%%%%%%%%%%%%%%%%%%%%%%%%%%%%%%%%%%%%%%%%%%%%%%%%%%%%%%%%%%%%%%%%%%%%%%%%
\section{Discussion}
\label{sec:qap_discussion}

The QAP is notoriously hard to solve, and currently known exact methods start to
take prohibitively long already for slightly more than $20$ objects, i.e., we
could barely solve the problem exactly for $5\times 5$ arrays. Fortunately, the
literature on QAP heuristics is rich, as many problems in operations research
can be modeled as QAPs. Here we used one such heuristic to identify the
potential of the MLP-QAP-relation.

As our results show, however, even heuristic algorithms are too slow to deal
with chips of dimensions larger than $12 \times 12$, and although we could
design a $20 \times 20$ chip with a QAP heuristic within a day, we have to keep
in mind that this would still be a very small part of bigger problem as real
microarray dimensions range from $200 \times 200$ up to $1164 \times 1164$.

For this reason, we restricted our experiments to such small chips and QAP
heuristics that could handle the problems within a few minutes. Up to now,
finding exact solutions even to these small microarrays seems to be an
incredible hard task. We mention here experiments conducted by Dr. Peter Hahn,
who used two branch-and-bound algorithms to solve some problem instances from
Table \ref{tab:graspr_greedy_bl}. With RTL-2 \citep{Adams}, it was possible to
find two solutions with total border length of $1\,652$ for a selected
$6\times 6$ chip, being only $1.43\%$ better than the solution found with GRASP-
PR ($1\,676$), although it took RTL-2 about $6.5$ hours, in contrast with the
less than 3 seconds needed by GRASP-PR. A lower bound calculation for the same
problem resulted in $1\,624$, so the RTL-2 solution is only $1.69\%$ higher,
while the gap to the GRASP-PR solution is about $3.10\%$.

For another selected problem of dimension $7\times 7$, Dr. Hahn found one
solution with border length $2\,290$ using RTL-1 \citep{Hahn1998}, being about
$1.72\%$ better than the solution found by GRASP-PR ($2\,330$), although it took
RTL-1 some 29 hours, in contrast with the less than 7 seconds needed for the
GRASP-PR run. The results obtained with exact QAP solvers give an idea of how
hard the quadratic assignment problem actually is, and show that the results
with GRASP-PR are a good compromise when time is limited.

Improved results for several selected problem instances from Tables
\ref{tab:graspr_greedy_bl} and \ref{tab:graspr_greedy_ci} were also reported by
Chris MacPhee using GATS, a hybrid genetic / tabu search algorithm, although
these results were obtained on a number of large memory SMP machines, each
having 144 processors and 576 GB of global memory. The latest results for these
selected problems are available online at \url{
http://gi.cebitec.uni-bielefeld.de/assb/chiplayout/qap}.

\subsection{Alternatives}

It is clear that, because of the large number of probes on industrial
microarrays, it is not feasible to use GRASP-PR (or any other currently
available QAP method) to design an entire microarray chip. However, we showed
that it is certainly possible to use it on small sub-regions of a chip, which
opens up the way for two alternatives.

First, the QAP approach could be used combined with a partitioning algorithm
such as those discussed in Chapter \ref{ch:part} to the design the smaller
regions that result from the partitioning. This, however, does not seem
promising because, as we will see later, a partitioning is a compromise in
solution quality, and level of partitioning required to achieve the dimensions
supported by the QAP approach is too high.

It is interesting to extrapolate the times shown on Table
\ref{tab:graspr_greedy_bl} to predict the total time that would be required to
design the layout of commercial microarrays, if we were to combine GRASP-PR with
a partitioning algorithm. If the partitioning produced $6\times 6$ regions,
$37\,636$ sub-regions would be created from the $1164\times 1164$ Affymetrix
Human Genome U133 Plus 2.0 GeneChip array, one of the largest Affymetrix chips.
Since each sub-region takes around 3 seconds to compute with GRASP-PR, the total
time required for designing such a chip would be a little over 31 hours
(ignoring the time for the partitioning itself).

If the partitioning produced $12\times 12$ regions, $9\,409$ sub-regions would
be created and, at 2.4 minutes each, the total time would be more than 16 days.
This is probably prohibitive, although it is certainly possible to reduce the
time of each GRASP-PR execution by running it on faster machines or run them in
parallel.

A better alternative is to use the QAP approach to improve an existing layout,
iteratively, by relocating probes inside a defined region of the chip, in a
sliding-window fashion. Each iteration of this method would produce an instance
of a QAP whose size equals the number of spots inside the window.The QAP
heuristics could then be used to check whether a different arrangement of the
probes inside the window can reduce the conflicts. For this approach to work,
however, we also need to take into account the conflicts due to the spots around
the window. Otherwise, a new layout with less internal conflicts could be
achieved at the expense of increasing conflicts on the borders of the window.

A simple way of preventing this problem is to solve a larger QAP instance
consisting of the spots inside the window as well as those in a layer (of three
spots) around it. The spots outside the window obviously must remain unchanged,
and that can be done by fixing the corresponding elements of the permutation
$\pi$. Note that there is no need to compute $f_{ij}$ if spots $i$ and $j$ are
both outside the window, nor $d_{kl}$ if probes $k$ and $l$ are assigned to
spots outside the window.

