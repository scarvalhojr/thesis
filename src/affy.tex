%%%%%%%%%%%%%%%%%%%%%%%%%%%%%%%%%%%%%%%%%%%%%%%%%%%%%%%%%%%%%%%%%%%%%%%%%%%%%%%%
\chapter{Analysis of Affymetrix Microarrays}
\label{ch:affy}
%%%%%%%%%%%%%%%%%%%%%%%%%%%%%%%%%%%%%%%%%%%%%%%%%%%%%%%%%%%%%%%%%%%%%%%%%%%%%%%%


General physical structure of GeneChip arrays. Control and special probes,
checkerboard patterns on the borders, empty spots.

Probe pairs: perfect match (PM) and mismatch (MM) probes. Rows of PM and MM
probes on the chip.

As we mentioned in Chaper~\ref{ch:affy}, all GeneChip arrays that we know of can be
asynchronously synthesized in 74~steps with the standard Affymetrix deposition sequence
(18.5 cycles of TGCA).

This suggests that only sub-sequences of this sequence
can be used as probes on Affymetrix chips.
\citet{Rahmann2006SubsequenceCombinatorics} shows that this
covers about 98.45\% of all 25-mers, however, it seems that Affymetrix uses
an even more restrictive probe selection criterion. This is because 
GeneChip probes
always appear in pairs, with the perfect match (PM) and the mismatch (MM) probes being located
next to each other (in alternating rows of PM and MM probes),
and there is evidence that the embeddings of these probes are aligned in such a way
that only the middle bases are not aligned.
