%%%%%%%%%%%%%%%%%%%%%%%%%%%%%%%%%%%%%%%%%%%%%%%%%%%%%%%%%%%%%%%%%%%%%%%%%%%%%%%%
\chapter{Re-embedding}
\label{ch:reembed}
%%%%%%%%%%%%%%%%%%%%%%%%%%%%%%%%%%%%%%%%%%%%%%%%%%%%%%%%%%%%%%%%%%%%%%%%%%%%%%%%

Once the probes have been placed, conflicts can be further reduced by
re-embedding the probes without changing their locations. All
re-embedding algorithms presented in this section are based on the
Optimum Single Probe Embedding (OSPE) introduced by \citet{Kahng2002}.
OSPE is a dynamic programming for computing an optimum embedding of a
single probe with respect to its neighbors, whose embeddings are
considered as fixed. The algorithm was originally developed for border
length minimization but here we present a more general form designed
for the conflict index model \citep{Carvalho2006}.

\section{Optimum Single Probe Embedding}
\label{sec:reembed_ospe}

The OSPE algorithm can be seen as a special case of a global alignment
between a probe sequence $p$ of length $\ell$ and the deposition
sequence $N$ of length $T$, disallowing mismatches and gaps in $N$.
We assume that $p$ is placed at spot $s$, and that we know the
embeddings of all probes placed at spots near $s$.

The optimal embedding of $p$ into $N$ is built by determining the
minimum cost of embedding a prefix of $p$ into a prefix of $N$: We use
an $(\ell + 1) \times (T + 1)$ matrix $D$, where $D[i,t]$ is defined
as the minimum cost of an embedding of $p[1..i]$ into $N[1..t]$. The
cost is the sum of conflicts induced by the embedding of $p[1..i]$ on
its neighbors, plus the conflicts suffered by $p[1..i]$ because of the
embeddings of its neighbors.

We can compute the value for $D[i,t]$ by looking at two previous
entries in the matrix: $D[i,t-1]$ and $D[i-1,t-1]$. The reason is that
$D[i,t]$ is the minimum cost of embedding $p[1..i]$ up to the
$t$-th synthesis step, which can only be obtained from the previous
step ($t-1$) by either masking or unmasking spot~$s$ at step~$t$.

If $s$ is productive at step $t$, base $N_t$ is appended to
$p[1..i-1]$; this is only possible if $p[i]=N[t]$. In this case a cost
$U_t$ is added for the risk of damaging probes at neighboring spots
$s'$. We know that $p[1..i-1]$ can be embedded in $N[1..t-1]$ with
optimal cost $D[i-1,t-1]$.  Hence, the minimum cost at step $t$, if
$s$ is productive, is $D[i-1,t-1] + U_t$.  According to the conflict
index model,
\[
U_t := \sum_{\substack{s'\text{: neighbor}\\\text{of } s}}
  \Ind{\eps_{k(s'),t}=0}
  \cdot \omega(\eps_{k(s')},t)
  \cdot \gamma(s',s).
\]


If $s$ is masked at step $t$, no base is appended to $p[1..i]$, but a
cost $M_{i,t}$ must be added for the risk of damaging $p$ (by light
directed at neighboring spots $s'$). Since $D[i,t-1]$ is the minimum
cost of embedding $p[1..i]$ in $N[1..t-1]$, the minimum cost up to step
$t$, if $s$ is unmasked, is $D[i,t-1] + M_{i,t}$.

Note that $M_{i,t}$ depends on the number of bases $p$ already
contains (that is, on $i$): Each unmasked neighboring spot $s'$
generates a conflict on $p$ with cost $\gamma(s,s') \cdot c \cdot
\exp[\theta\cdot (1+\min\{i,\ell-i\})]$, in accordance with
(\ref{eq:pos_mult})--(\ref{eq:b_ell}). Thus
\[
M_{i,t} := c \cdot \exp[\theta\cdot (1+\min\{i,\ell-i\})] \cdot
\sum_{\substack{s'\text{: neighbor}\\\text{of } s}}
\Ind{\eps_{k(s'),t}=1}  \cdot \gamma(s,s').
\]

Finally, $D[i,t]$ is computed as the minimum cost of the possible
actions,
%%
\[
D[i,t] := \begin{cases}
  \min \{\, D[i,t-1] + M_{i,t},\;  D[i-1,t-1] + U_t \,\}
  & \text{ if $p[i]=N[t]$,}\\
  D[i,t-1] + M_{i,t}
  & \text{ if $p[i]\neq N[t]$.}
  \end{cases}
\]

The first column of $D$ is initialized as follows: $D[0,0] = 0$ and
$D[i,0] = \infty$ for $0 < i \leq \ell$, since no probe of length
$\ell > 0$ can be embedded into an empty deposition sequence. The
first row is initialized by setting $D[0,t] = D[0,t-1]+M_{0,t}$ for
$0<t\leq T$.

If we assume that costs $U_t$ and $M_{i,t}$ can be computed in
constant time, the time complexity of the OSPE algorithm is $O(\ell
T)$ since there are $O(\ell T)$ entries in $D$ to compute. The algorithm can
be rather time-consuming in the general form presented here, since we
have to look at the embeddings of up to 48 neighbors around~$s$.
Naturally, it runs much faster for border length minimization, since
there are only four neighbors, and there are neither
position-dependent ($\omega$) nor distance-dependent ($\gamma$)
weights to compute. In any case, a few optimizations significantly
reduce the running time.  For instance, in each row, only the columns
between the left-most and the right-most embedding of $p$ in $N$ need
to be computed.

Once $D$ is computed, the minimum cost is $D[\ell,T]$, and an
optimal embedding of $p$ into $N$ can be constructed by tracing a path from
$D[\ell,T]$ back to $D[0,0]$, similarly to the procedure used to build an
optimal global alignment.  This takes $O(T)$ time.



\section{Re-embedding algorithms}
\label{sec:reembed_alg}

The OSPE algorithm is the basic operation of several post-placement
optimization algorithms: Greedy, Batched Greedy and Chessboard
\citep{Kahng2002}, and Sequential~\citep{Kahng2003a}. Their main difference
lies in the order in which the probes are re-embedded.

Since OSPE never increases the amount of conflicts in the region
around the re-embedded probe, optimization algorithms can execute
several re-embedding operations without risk of worsening the current
solution. Moreover, probes can be re-embedded several times since new
improvements may be possible once neighbors are changed.
In fact, the following
algorithms work in repeating cycles of optimization until no more
improvements are possible (when a local optimal solution is found), or
until improvements drop below a given threshold.
  
The Greedy algorithm uses OSPE to compute, for each spot of the chip,
the maximum reduction of border conflicts achievable by optimally
re-embedding its probe. It then selects a spot $s$ with the highest
gain (reduction of conflicts) and re-embeds its probe optimally,
updating the gains of affected neighboring spots.

A faster version of this algorithm, called Batched Greedy, pre-selects
several spots for re-embedding and thus sacrifices its greedy nature
by postponing the update of gains.

The Chessboard optimization is based on the fact that a chip can be bi-colored
like a chessboard, in such a way that the embeddings of probes located on
white spots are independent of those placed on black spots (with respect to
border length), and vice-versa. The Chessboard uses this coloring to alternate
the optimal re-embedding of probes located on black and white spots.

The Sequential optimization is the simplest algorithm among the four.
It proceeds spot by spot, from top to bottom, from left to right,
re-embedding each probe optimally. Once the end of the array is
reached, it restarts at the top left for the next iteration.

Surprisingly, the Sequential algorithm achieves the greatest reduction
of border conflicts with a running time comparable to Batched Greedy,
the fastest among the four.  All re-embedding algorithms mentioned
here were initially developed for border length minimization, but they
can all be applied to the conflict index model as well. For the
Chessboard optimization, $4\times 4=16$ colors must be used
instead of~$2$.
