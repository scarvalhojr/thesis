%%%%%%%%%%%%%%%%%%%%%%%%%%%%%%%%%%%%%%%%%%%%%%%%%%%%%%%%%%%%%%%%%%%%%%%%%%%%%%%%
\chapter{The Shortest Deposition Sequence Problem}
\label{ch:scs}
%%%%%%%%%%%%%%%%%%%%%%%%%%%%%%%%%%%%%%%%%%%%%%%%%%%%%%%%%%%%%%%%%%%%%%%%%%%%%%%%

Several interesting computational problems emerge in the design and production
of DNA microarrays. The shortest common supersequence (SCS) problem is one such
a problem, which arises during the synthesis of probes on the array. It is an
classical problem in computer science that has already been proved to be
NP-complete. The aim of this report is to analyze the feasibility of finding
shortest common supersequences for the production of DNA microarrays. Several
different strategies are considered. Because of time and space constraints, we
propose a branch-and-bound depth-first search. This strategy relies on computing
upper and lower bounds on the length of the SCS to prune the search space. A
computer program was implemented to analyze its performance in practice. The
implementation is described in detail, along with several attempts to reduce its
running time. Finally, we describe some experiments and conclude with an
evaluation of the implementation as well as the feasibility of such an approach.

%%%%%%%%%%%%%%%%%%%%%%%%%%%%%%%%%%%%%%%%%%%%%%%%%%%%%%%%%%%%%%%%%%%%%%%%%%%%%%%%
\section{Introduction}
\label{sec:scs_intro}

In the arrays manufactured by Affymetrix, the probe sequences are
synthesized on the chip, in parallel, nucleotide by nucleotide. The
process consists of a series of steps. In each step the same
nucleotide is appended to all probes located on selected spots. This
selection occurs by exposure to light, with a mask that is placed over
the chip to ensure that only the appropriate spots are activated to
receive the nucleotide. After that, a solution containing the
nucleotides is flushed on the chip and allowed to hybridize.

The sequence of nucleotides used in the synthesis process is called
the \textit{deposition sequence}. Clearly, each probe is a
(non-contiguous) subsequence of the deposition sequence, which is, in
turn, a supersequence of the set of probes. It is common to use a
deposition sequence that consists of repetitions of \Seq{ACGT}. This
sequence is used because of its simplicity and relative effectiveness
\cite{Rahmann2004}.

Ideally, the deposition sequence should be as short as possible so
that the number of steps is reduced and, consequently, the
manufacturing time is shortened. Moreover, the masking process is
subject to errors, i.e., there is a risk of having the light shining
over a spot that was not supposed to be activated. Hence, if the
number of steps is reduced, the chance of unwanted spot activation is
also reduced, and the overall quality of the chip is improved.

The issue of accidental activation can also be diminished by carefully
arranging the probes on the chip. It is evident that the spots more
likely to be affected by stray light are those which lie immediately
next to an unmasked spot. More precisely, the chances are higher on
the \emph{borders} shared with an unmasked spot. For this reason, the
risk of errors can be reduced by minimizing the number of borders
shared by masked and unmasked spots in each step of the synthesis
process.

This report, however, will concentrate on minimizing the length of the
deposition sequence. More precisely, we are interested in finding the
shortest supersequence of a given set of sequences such that all
probes of a microarray chip can be synthesized. This is, of course, a
well-known problem in the computer science literature and is usually
referred as the shortest common supersequence (SCS) problem. It was
shown to be NP-complete for strings over an alphabet of size greater than or equal to 2.
\cite{Raiha1981}. For this reason, several heuristics have been
devised to compute approximate solutions more efficiently ---
see \citet{Fraser1995} for a survey. But, to this day,
finding an exact solution seems to be limited to small sets of
sequences and reduced alphabet sizes.

The objective of this report is, therefore, to analyze the feasibility
of computing a solution to the SCS problem in the context of the
production of DNA microarrays.

Formally we have a set of $n$ sequences $\mathcal{P} = \{ p_{1},
p_{2}, p_{3}, \ldots, p_{n} \}$, drawn from the alphabet $\Omega$,
whose size is $z = | \Omega |$. The length of any sequence $p_{i}$ is
$m = |p_{i}|$ (for simplicity, all sequences are assumed to have the
same length, but this assumption is not necessary).  Given this input,
we want to find $s$, the shortest common supersequence for the set
$\mathcal{P}$. The DNA chip production setting constrains these
variables to the values shown in table \ref{tab:constraints}.

\begin{table}
\centering
\begin{tabular}{|l|c|} \hline
Number of sequences: & $10,000 \leq n \leq 1,000,000$ \\ \hline
Length of sequences: &$20 \leq m \leq 30$ \\ \hline
Alphabet: & $\Omega = \{\tA,\tC,\tG,\tT\}$ \\ \hline
Alphabet size: & $z = | \Omega | = 4$ \\ \hline
\end{tabular}
\caption{Variable constraints in the DNA chip production setting.}
\label{tab:constraints}
\end{table}

%%%%%%%%%%%%%%%%%%%%%%%%%%%%%%%%%%%%%%%%%%%%%%%%%%%%%%%%%%%%%%%%%%%%%%%%%%%%%%%%
\section{Alternatives}
\label{sec:scs_alternatives}

Several approaches may be tried in order to find an exact solution to
the shortest supersequence problem. Naively, one can generate all
possible sequences with length $k$, checking whether each of them is a
viable supersequence. The initial value of $k$ can be set to $m$ (the
length of the sequences) since a supersequence must be, at least,
$m$~characters long. If no supersequence is found with $k$ characters,
the value of $k$ is increased and all sequences of this new length are
generated and examined. The value of $k$ denotes the length of the
shortest common supersequence as soon as a supersequence is found.

Clearly, $m$ is not a good initial value for $k$. The shortest
supersequence of a typical DNA chip can be expected to be between 55
and 95 characters long (see \cite{Rahmann2003}). Fortunately, reasonably
good lower bounds on the length of the SCS can easily be computed and
the search can start with a higher value for $k$.

With the methods described in section \ref{sec:scs_lbound}, one can expect
to find a lower bound of around 60 characters for a typical set of $n
= 10,000$ probes of length $m = 25$. Starting from $k = 60$ (and not
$k = 25$) alone saves a significant amount of time. But this is not
enough. For $k = 60$, about $1.33\cdot 10^{36}$ candidate sequences
are generated. This approach is obviously not feasible since, for each
candidate, we still need to check whether it is a supersequence or
not.

Efficient algorithms for the SCS problem can be found in the computer
science literature. Most of them are based on dynamic programming ---
such as those developed by Itoga \cite{Itoga1981}, Foulser at al.
\cite{Foulser1992} and the improvements suggested by Fraser
\cite{Fraser1995}. However, they were not developed for problem sets in
the order we are investigating here. In fact, they are not feasible in
this context due to their $O(m^{n})$ space complexity (recall that the
number of sequences $n$ is usually in the order of $10^{5}$). In
practice these algorithms can only be used to solve problem instances
with small $n$.

The only viable approach to compute an exact solution to the SCS problem of such
magnitude seems to be a branch-and-bound search. The reason is that its space
complexity is merely $O(n m)$ for simple implementations (and this memory is
used only for storing the input sequences).

In practice, as it will be demonstrated in the next section, the approach taken
here employs techniques that speed up the algorithm at the expense of greater
memory requirements. This approach is similar to the one used by
\citet{Fraser1995}, although he worked on much smaller problem instances (the number of
sequences was not greater than 24).

\subsection{Searching for the SCS}

As stated earlier, the approach adopted here to solve the SCS problem for the
DNA chip production setting is based on a branch-and-bound search strategy.
Essentially, this approach is an improvement on the naive algorithm described in
the previous section.

Consider what happens when the naive algorithm searches for a SCS. For
the sake of simplicity, suppose the alphabet from which the sequences
are draw consists of only three letters, say $\Omega = \{A, B, C\}$.
As $k$ increases from 1 to 3, the following sequences are generated,
in this order (increasing $k$, from left to right):

\begin{tabbing}
$k = 1 \Rightarrow$ \= $A$, $B$, $C$ \\
$k = 2 \Rightarrow$ \= $AA$, $AB$, $AC$, $BA$, $BB$, $BC$, $CA$, $CB$, $CC$ \\
$k = 3 \Rightarrow$ \= $AAA$, $AAB$, $AAC$, $ABA$, $ABB$, $ABC$, $ACA$, $\ldots$, $CCC$
\end{tabbing}

If no supersequence of length 3 is found, three sequences are generated when
$k = 1$, nine when $k = 2$ and 27 when $k = 3$, totaling 39 candidate sequences.
These sequences can be arranged in a (complete) tree $\mathcal{T}$ of height
$h = 3$ in such a way that each candidate sequence is represented by a path in
$\mathcal{T}$ (and each path in $\mathcal{T}$ yields a candidate). Each node has
precisely three children, one for each possible letter of the alphabet, and the
root node represents an empty sequence.

It is not difficult to see that the naive algorithm examines each node of
$\mathcal{T}$ in a \textit{breadth-first} fashion. It first investigates all
nodes in the first level of $\mathcal{T}$ (directly reached from the root node),
then all nodes of the second level (reached from first-level nodes), and finally
all nodes of the third level.

Alternatively, the nodes of $\mathcal{T}$ could be explored in a
\textit{depth-first} fashion, which searches ``deeper'' in the graph whenever
possible \cite{Cormen2001}. In this way, the sequences are visited in
the following order: $A$, $AA$, $AAA$, $AAB$, $AAC$, $AB$, $ABA$, $ABB$, $ABC$,
$AC$, $ACA$, $ACB$, $ACC$, $B$, $BA$, $BAA$, $\ldots$, $CCC$.

Changing from a breadth-first to a depth-first search does not make the
algorithm any better. The key point is that it can be more easily combined with
a branch-and-bound strategy to produce an efficient way of exploring the search
space.

\subsection{A Branch-and-Bound Strategy}

A branch-and-bound strategy means that, before exploring a node
further down, we check whether the node has a chance of leading to a
better solution than the best solution found so far
\cite{Horowitz1996}. If it does not, the node is ignored and the
search proceeds to the next node. The implications of this definition
are two-fold.

First, it implies that we already have a solution, although we are
looking for a better one --- or, ultimately, the best solution.
Moreover, we need an \textsl{initial} solution, even before the search
starts. An initial solution means one proper supersequence of the set,
preferably, one that is relatively short and that can be found
relatively quickly. For this purpose, we can use the heuristics
algorithms described in section \ref{sec:scs_ubound}. These algorithms are
able to quickly produce an approximate solution to the SCS. (In
practice, we can run several of those methods and pick the shortest
sequence among them.) This approximate solution is an upper bound that
defines the limits of the search-space that needs to be explored.
During the search, we keep track of the best solution found so far,
i.e., the shortest known supersequence of the set. The shorter it is,
the more branches of the tree can be skipped.

Second, we need a way of checking whether a node can lead to a better
solution or not. More precisely, we must be able to predict the length
of the shortest supersequence that can be found from the node, i.e. a
lower bound on the length of any supersequence reachable from it.
Again, an exact answer is not necessary, but only a good and quickly
computed estimate. For this purpose we can use the heuristic
algorithms described in section \ref{sec:scs_lbound}. These algorithms
compute the minimum length that a sequence must have in order to be a
proper supersequence of the set; in other words, they give a lower
bound on the length of the supersequence.

Note that, when the search is at a node $x$ of the tree, its corresponding
sequence $d_{x}$ is a prefix of a set of candidate sequences. Moreover, for each
sequence $p_{i}$ in the set $\mathcal{P}$, $d_{x}$ is a supersequence of a
(possibly empty) prefix of $p_{i}$. Let $c_{i}$ be the longest prefix of $p_{i}$
which is a subsequence of $d_{x}$, and $r_{i}$ be the remainder of $p_{i}$ in
such a way that $p_{i}$ is a concatenation of $c_{i}$ and $r_{i}$. In order to
be a proper supersequence of $\mathcal{P}$, $d_{x}$ must be extended with a
suffix $u$ in such a way that it ``consumes'' the remainders of each sequence.
In other words, $u$ must be a supersequence of the set
$\mathcal{R} = \{r_{1}, r_{2}, \ldots, r_{n}\}$. Naturally, the algorithms of
section \ref{sec:scs_lbound} can again be used to estimate the minimum length of
$u$. And since we know the length of $d_{x}$, we can estimate the minimum length
of any sequence that has $d_{x}$ as a prefix must have in order to be a proper
supersequence of $\mathcal{P}$. At this point, if this minimum length is already
greater than the length of the shortest known supersequence, the node can safely
be ignored.

In this way, the algorithm proceeds with the goal of finding a
sequence that is shorter than the best supersequence already found
(known as the search's \textit{bounding condition}). And whenever a
better solution is encountered, the goal is updated and, hopefully,
more parts of the tree can be pruned off.

This is precisely the approach taken here, a branch-and-bound depth-first
search. Note that a branch-and-bound strategy could also be used in conjunction
with a breadth-first search. However, doing so would require keeping track of
the branches which are still ``alive'' --- i.e. the nodes in the current level
of the tree that need to be further investigated --- and this would consume an
enormous amount of memory as the search reached deeper levels of the tree.

A depth-first search, on the other hand, does not require such
bookkeeping. Each child of a node is reached by a different letter of
the alphabet, and can be explored in \emph{any} order, e.g., their
alphabetical order, and the order can be different for each node. When
the search backtracks to a certain node, the next child is selected.
When a node is skipped, the search backtracks until it finds a node
whit an estimate that is greater than the shortest sequence already
found.

%%%%%%%%%%%%%%%%%%%%%%%%%%%%%%%%%%%%%%%%%%%%%%%%%%%%%%%%%%%%%%%%%%%%%%%%%%%%%%%%
\section{Upper bound algorithms}
\label{sec:scs_ubound}

In this section, two heuristic algorithms used to compute an approximate
solution to the shortest common supersequence  problem will be briefly
described: \textsc{MajorityMerge} and \textsc{AlphabetCycle}. These algorithms
are used to set an upper bound on the length of the SCS for the branch-and-bound
search algorithm described in details in section \ref{sec:scs_implementation}. Refer to
\cite{Fraser1995} for other approximation algorithms.

\subsection{AlphabetCycle}
\label{sec:scs_acycle}

The \textsc{AlphabetCycle} algorithm is essentially the trivial
\textsc{AlphabetLeftmost} algorithm analyzed in more detail by Rahmann in
\cite{Rahmann2003}. Let $\lambda$ be any permutation of letters of the alphabet
$\Omega$. If $m$ is the length of the longest input sequence and $s$ is an
m-fold repetition of $\lambda$, $s$ is a supersequence of the set. Moreover, it is
a $|\Omega|$-factor approximation of the SCS. The sequence $s$ can be further
improved by considering the left-most embedding of the input sequences in $s$
(where ``useless'' characters are removed).

This algorithm can be easily implemented in the following way. First set $s$ to
the empty sequence. Given a permutation $\lambda$, the algorithm proceeds by
constructing $s$ in a series of steps. At each step, it takes a character $c$ of
$\lambda$. Then it removes the first character of all input sequences that have
$c$ as a prefix, and append $c$ to $s$. If no sequence has $c$ as a prefix at a
given step of the algorithm, the character is simply ignored and not appended to
$s$. The character $c$ assumes values from $\lambda$ in a cyclical way. The
algorithm terminates when all sequences have been reduced to empty sequences.

According to Rahmann and to our own empirical results, this algorithm is hard to
be outperformed \cite{Rahmann2003}. The choice of the permutation $\lambda$ is not
really important but if the alphabet is small, it is worth trying all possible
permutations of $\Omega$ and select the best result.

\subsection{MajorityMerge}
\label{sec:scs_mmerge}

This algorithm was proposed by Jiang and Li \cite{Jiang1995}, and it
can be seen as a variation of the \textsc{AlphabetCycle}. The idea is
that the supersequence is also built in a series of steps until all
input sequences have been reduced to empty sequences.

The choice of the character $c$, however, is not taken from a fixed
permutation. Instead, at each step, the algorithm examines the first
character of all sequences and sets $c$ to the most frequent one. Ties
are broken arbitrarily or other measure is used such as considering
the average length of the remaining sequences.

MajorityMerge typically performs much worse on typical
microarray-scale problems than AlphabetCycle \cite{Rahmann2003}.

%%%%%%%%%%%%%%%%%%%%%%%%%%%%%%%%%%%%%%%%%%%%%%%%%%%%%%%%%%%%%%%%%%%%%%%%%%%%%%%%
\section{Lower Bound Algorithms}
\label{sec:scs_lbound}

This section will briefly describe some heuristic algorithms that can be used to
compute a lower bound on the length of the SCS.

\subsection{Simple algorithm}
\label{sec:scs_simple}

Perhaps the most simple way of computing a lower bound on the length of the SCS
is to simply take the length of the longest sequence in the set. If the longest
sequence in $\mathcal{P}$ has length $x$, the SCS must be, at least, $x$
characters long. Clearly, when all sequences have the same length (like in the
context this work is focused on, where all sequences have length $m$), this
result is not really interesting. On the other hand, if the sequences are short,
have different lengths and are restricted to a small alphabet, this
easy-to-compute algorithm can be an option to consider.

\subsection{Counting occurrences of single letters}
\label{sec:scs_singleletters}

A stronger result is achieved by finding the maximum number of
occurrences of the letter $c$ over all sequences, $\mathcal{N}(c)$,
for all letters of the alphabet. Obviously, the shortest common
supersequence must have, at least, $\mathcal{N}(c)$ occurrences of $c$
(figure \ref{fig:countingSingleLetters}).

\begin{figure}
\centering
\begin{tabular}{|r|c|c|c|c|} \hline
$x =$             & A & B & C & Sum\\ \hline \hline
CABBABAC          & 3 & 3 & 2 & 8 \\
CCABBABC          & 2 & 3 & 3 & 8 \\
BBBBAACC          & 2 & 4 & 2 & 8 \\ \hline
$\mathcal{N}(x)=$ & 3 & 4 & 3 & 10 \\ \hline
\end{tabular}
\caption{\textbf{Counting occurrences of single letters.}
A lower bound on the length of the SCS is computed for a set of three sequences
of length 8. The alphabet is $\Omega = \{A, B, C\}$. After the number of $A$s,
$B$s and $C$s are counted, the maxima over all sequences are taken and summed
up. The SCS must have, at least, 3 $A$s, 4 $B$s and 3 $C$s. Its length,
therefore, cannot be shorter than 10.}
\label{fig:countingSingleLetters}
\end{figure}

\subsection{Counting pairs and triples}
\label{sec:scs_pairstriples}

The previous algorithm can be naturally extended to count occurrences of pairs
or even triples instead of single letters, and the same reasoning can be used to
compute a lower bound as illustrated in figure \ref{fig:countingPairs}.

It should be clear that the algorithm of figure \ref{fig:countingPairs} produced
a tighter lower bound (11) than the one produced with the method described in
figure \ref{fig:countingSingleLetters} (10). That is, therefore, a better
result. While the previous method proved that the SCS must have at least 10
characters, figure \ref{fig:countingPairs}  goes even further by showing that in
fact, it can be no shorter than 11 characters long.

It might be somewhat intuitive to think that this algorithm is likely to produce
better results than simply counting single letters because its prediction is
based on ``more information''. However, much to one�s surprise, this method
rarely produces better results in the context of DNA chip production. In fact,
the example of figures \ref{fig:countingSingleLetters} and
\ref{fig:countingPairs} were carefully designed to produce a tighter bound with the
second method. Furthermore, this method clearly has a higher cost in terms of
time complexity. While the method of section \ref{sec:scs_singleletters} can be
implemented in linear time, this method has $O(n^{2})$ time complexity.

\begin{figure}
\centering
\begin{tabular}{|r|c|c|c|c|c|c|c|c|c|c|} \hline
$x =$              & AA & AB & AC & BA & BB & BC & CA & CB & CC & Sum \\ \hline \hline
CABBABAC           & 3  & 4  & 3  & 5  & 3  & 3  & 3  & 3  & 1  & 28 \\
CCABBABC           & 1  & 4  & 2  & 2  & 3  & 3  & 4  & 6  & 3  & 28 \\
BBBBAACC           & 1  & 0  & 4  & 8  & 6  & 8  & 0  & 0  & 1  & 28 \\ \hline
$\mathcal{N}(xy)=$ & 3  & 4  & 4  & 8  & 6  & 8  & 4  & 6  & 3  & 46 \\ \hline
\end{tabular}
\caption{\textbf{Counting pairs.} This is the
same example shown in figure \ref{fig:countingSingleLetters}. Each sequence has
length 8 which gives room for 28 possible pairs. After the number of occurrences
of each possible pair are counted, the maxima over all sequences are taken and
summed up. The SCS must have, at least, 3 $AA$s, 4 $AB$s, 4 $AC$s and so on,
totaling 46 distinct pairs. Since it must accommodate 46 pairs, the SCS can be
no shorter than 11 (a sequence of length 10 has only ${10 choose 2}=45$
pairs).}
\label{fig:countingPairs}
\end{figure}

\subsection{Looking for Better Estimations}

As it will become clear in the next section, computing a good lower bound on the
length of the SCS is key to the success of the approach developed in this work.
After some of the previous algorithms had been tried, it was obvious that a
better method was needed. In the pursuit of such a method, some relations were
investigated, like the following one:

\begin{equation}
|w|_{xy} + |w|_{yx} = |w|_{x} \times |w|_{y}
\label{eq:simple}
\end{equation}

This relation is valid for any $w \in \Omega^{*}$ and
$x, y \in \Omega, x \neq y$. The notation $|w|_{xy}$ refers to the number of
occurrences of the pair $xy$ in $w$.

It is not difficult to see that this relation holds for any sequence, and hence
it must also hold for the supersequence. The question is: what happens when we
take the maximum number of occurrences of single letters and pairs over the set
of sequences?

\begin{equation}
\mathcal{N}(xy) + \mathcal{N}(yx) \; ? \; \mathcal{N}(x) \times \mathcal{N}(y)
\label{eq:simpleDerived}
\end{equation}

Here $\mathcal{N}(x)$ is the maximum number of occurrences of the letter $x$ in
the input sequences for $x \in \Omega$. Similarly, $\mathcal{N}(xy)$ is the
maximum number of occurrences of the pair $xy$ for $x, y \in \Omega$. Again, the
SCS must have at least $\mathcal{N}(x)$ occurrences of $x$, and at least
$\mathcal{N}(xy)$ occurrences of the pair $xy$.

Given the input sequences, we can easily compute $\mathcal{N}(x)$ and
$\mathcal{N}(xy)$ for all $x, y \in \Omega$. It seemed intuitive that a
``greater than'' would be found in relation (\ref{eq:simpleDerived}) since
counting pairs would ``carry more information'' than counting single letters. If
this was the case, we could estimate the length of the SCS in the following way.
First, we would compute $\mathcal{N}(x)$ and $\mathcal{N}(x)$ for all
$x, y \in \Omega$. Then, with these values in hand, we would create several
relations in the form of (\ref{eq:simpleDerived}) and increase the values on the
right-hand side of these relations (the values of $\mathcal{N}(x)$ and
$\mathcal{N}(x)$) until we had equalities as in (\ref{eq:simple}).

However, the intuition turned out to be wrong. It is still not clear why this is
the case but, in practice, it was observed that the relation
(\ref{eq:simpleDerived}) have a ``less than'' or an ``equal'' sign in the majority of
the cases, which prevented it to be used to compute a tighter lower bound on the
length of the SCS.

Another interesting relation that seemed promising in the beginning was the
Cauchy inequality \cite{Salomaa2003} \cite{Mateescu2004}:

\begin{equation}
|w|_{y} \times |w|_{xyz} \leq |w|_{xy} \times |w|_{yz}
\label{eq:cauchy}
\end{equation}

Once more, the notations $|w|_{y}$ and $|w|_{xyz}$ refers to the number of
occurrences of the sequence $y$ and $xyz$ in $w$, respectively, for
$x, y, z \in \Omega^{*}$. It is clear that this inequality also holds when $x$,
$y$ and $z$ are not sequences but single letters ($x, y, z \in \Omega$. Now
consider a similar relation concerning the maximum number of occurrences of
single letters, pairs and triples over all sequences:

\begin{equation}
\mathcal{N}(y) \times \mathcal{N}(xyz) \; ? \; \mathcal{N}(xy) \times \mathcal{N}(yz)
\label{eq:cauchyDerived}
\end{equation}

Contrary to the case with relation (\ref{eq:simpleDerived}), there is no
intuitive notion to predict how it behaves in practice. However, if a ``greater
than'' was the case, we could estimate the length of the SCS by increasing the
values of $\mathcal{N}(yz)$ and $\mathcal{N}(xy)$ until we had a ``less than or
equal'' sign, agreeing with relation (\ref{eq:cauchy}).

But, unfortunately, this is not the case. When we check this relation in
real-world examples, we find a ``less than or equal'' sign, already in accordance
with the Cauchy inequality. Then again, it was not possible to conceive a better
method to compute a lower bound on the length of the SCS.

%%%%%%%%%%%%%%%%%%%%%%%%%%%%%%%%%%%%%%%%%%%%%%%%%%%%%%%%%%%%%%%%%%%%%%%%%%%%%%%%
\section{Implementation}
\label{sec:scs_implementation}

So far we have only outlined the approach used in developing a computer
program to search for the shortest common supersequence in the context of DNA
chip production. In this section, the actual implementation will be described in
more detail.

The program takes as input a text file with $n$ sequences of length $m$, one
sequence per line. Each sequence consists of a series of characters draw from a
fixed (and small) alphabet $\Omega$ which must be known beforehand. The
sequences are read and stored in memory.

As explained earlier, in order to start the search, an upper bound on the length
of the SCS is needed, and both \textsc{MajorityMerge} and \textsc{AlphabetCycle}
(described in section \ref{sec:scs_ubound}) are used for this purpose. In fact, the
\textsc{AlphabetCycle} is run with every possible permutation of the alphabet
and the best result is kept in a variable $\mathcal{U}$. (Since these algorithms
are relatively fast and run only once, their influence on the total running time
is negligible.) In fact, at any given time, $\mathcal{U}$ will store the length
of the shortest known supersequence.

Recall that the approach taken here is a
branch-and-bound depth-first search. This means that the program will look for
the SCS in a tree which represents all sequences that can be constructed with
the characters of $\Omega$.

The search starts from the root node and proceeds down the tree as far
as it can. At every node $x$, the program computes a lower bound $L_x$
on the length of the shortest supersequence that can be found from
$x$. For this purpose one algorithm of those described in section
\ref{sec:scs_lbound} is used (more on the topic later). Again, a node $x$
in this tree represents a sequence $d_{x}$ of $\Omega^{*}$. If $d_{x}$
is not a proper supersequence of the set, the program must search for
a supersequence further down the tree. However, the search proceeds to
a child node of $x$ only if $L_{x} + |d_{x}| < \mathcal{U}$. Otherwise
node $x$ is skipped and the search proceeds to a sibling node of $x$
or, if there is no sibling node of $x$ which has not been visited yet,
the search goes back to the parent node of $x$.

Whenever the search finds a sequence $d_{x}$ which is a supersequence
of the set, it knows that $d_{x}$ is shorter than the previously known
supersequence (otherwise the search would have never reached the
corresponding node). In this case, $\mathcal{U}$ is set to $|d_{x}|$.
Then, with a new value for $\mathcal{U}$, the search backtracks to a
node $y$ in the tree where $L_{y} + |d_{y}| < \mathcal{U}$.

The program proceeds in this way until the tree has been completely traversed.
In the end, the program can assure that the length of the shortest common
supersequence for the input sequences is $\mathcal{U}$ since it can be proven
that all nodes which have not been examined can not be a prefix of a
supersequence whose length is shorter than $\mathcal{U}$.

\subsection{Bounding condition}

With regard to the evaluation of a node (checking if it can lead to a
supersequence that is shorter than the one already found), several approaches
have been tried as described in section \ref{sec:scs_lbound}. Obviously, unlike the
initial upper bound estimation, we cannot afford to run all methods to choose
the best result because this estimation is done at every single node of the
search space. Therefore, one strategy must be chosen.

Initial experiments quickly revealed that the best alternative is to compute the
lower bound based solely on the number of occurrences of single characters.
The reasons are two-fold. First,
this strategy produced the best results in the majority of cases. Second, it is
significantly faster and consumes less memory than the others. The experiments
also showed that counting pairs to compute the lower bound produced tighter
bounds in only 10\% of the cases. However, the
extra costs could not compensate for the slightly tighter bounds. In other
words, although the tighter bounds meant more branches of the tree could be
pruned off, the time spent in estimating the lower bound and maintaining the
necessary data structures was greater than the time saved for not traversing
those branches.

\subsection{Computing lower bounds}

Recall that when the search is at one node
$x$ of the tree, the sequence $d_x$ represented by $x$ is a prefix of
a set of candidate sequences. Moreover, for each sequence $p_{i}$ in
the input set $\mathcal{P}$, $d_{x}$ is a supersequence of a (possibly
empty) prefix of $p_{i}$. Let $c_{i}$ be the longest such prefix and
$r_{i}$ be the remainder of $p_{i}$ in the same way defined earlier.
Recall that, in order to be a proper supersequence of $\mathcal{P}$,
$d_{x}$ must be extended with a suffix $u$ in such a way that it
``consumes'' the remainders of each sequence. In other words, $u$ must
be a supersequence of the set $\mathcal{R} = \{r_{1}, r_{2}, \ldots,
r_{n}\}$. This means that the value $L_{x}$ of a node is computed by
finding the SCS of $\mathcal{R}$.

As the search proceeds from one node to the other, the program keeps track of
the longest prefix of each sequence $p_{i}$ in $\mathcal{P}$ that is a
subsequence of $d_{x}$, with the help of a table containing $n$ index values,
$I_{1}$ to $I_{n}$, one for each input sequence. Before the search starts, the
index values are initialized to zero. When the search proceeds from a parent
node to one of its children, (incrementing the sequence $d_{y}$ with character
$c$ to produce $d_{x}$) the program examines every input sequence $p_{i}$ at the
position indicated by the corresponding index $I_{i}$. If $p_{i}[I_{i}]$
contains the character $c$, $I_{i}$ is incremented by 1. Otherwise $I_{i}$ is
left unchanged.

Clearly, when the search proceeds from the child node to its parent, a similar
procedure must also be executed to update the index values. But what may not be
so easy to realize is the fact that, with the index table alone, this mechanism
is not reversible.

In fact, the program needs to maintain a second table $R$, called reverse index
table, whose size is equal to $n \times m$. This table is updated whenever an
index $I$ is updated. If an index $I_{i}$ is incremented the program records the
length of the sequence $d_{x}$ in $R_{i}[I_{i}]$. This indicates that the
character at the position $I_{i}$ of $p_{i}$, $p_{i}[I_{i}]$, corresponds to the
character at the position $R_{i}[I_{i}]$ of $d_{x}$. Note that the sequence
$d_{x}$ will be a prefix of any sequence $d_{z}$ as the search progresses down
the tree and, therefore, $I_{i}$ will always point to a valid position.

Now, when the search goes from child node $x$ to parent node $y$, it is possible
to update table $I$ with the help of table $R$. The index $I_{i}$ is decremented
only if a) $p_{i}[I_{i}]$ is equal to the character $c$ corresponding to the
edge that is being traversed back and b) $R_{i}[I_{i}]$ is equal to $|d_{x}|$.

In this way, the program can quickly determine the set $\mathcal{R}$ for which
it needs to find the SCS. But this is not enough. The most critical part of the
search is, precisely, computing the lower bounds at every node of the tree.
Hence, it is not feasible to run the algorithm of section \ref{sec:scs_singleletters}
from scratch for every instance of $\mathcal{R}$. A cleverer way is needed.
Indeed, if we carefully observe how the search traverses the tree, we can note
that it is possible to compute the lower bound $L_{x}$ of node $x$ ``reusing''
information gained when computing $L_{y}$ for $y$, the parent node of $x$. This
is true due to the fact that very little changes are observed in the set
$\mathcal{R}$ when the search goes from nodes $y$ to $x$ and vice-versa.

Hence, the program also keeps track of, for each input sequence, the number of
occurrences of each letter of the alphabet, $N_{i}[c]$. This table is updated
together with the index table $I$. Whenever the index $I_{i}$ is incremented,
the number of occurrences of the character $p_{i}[I_{i}]$, $N[p_{i}[I_{i}]]$ is
decremented by 1, and vice-versa.

However, in order to compute the lower bound on the length of the SCS of
$\mathcal{R}$, the program needs to find the maximum value of $N_{i}[c]$ over
all sequences, $Max[c]$, for all letters of the alphabet. Fortunately, this
information can also be maintained as the search goes from one node to the
other.

When going from one node to its parent, if a particular $N_{i}[c]$ is
incremented and becomes greater than $Max[c]$, $Max[c]$ is updated.
Unfortunately, the reverse procedure is not so simple. In fact, when going from
one node to one of its children, the program must first update the $N_{i}[c]$
values for all sequences and all letters of the alphabet. Only after that, it
can update the $Max[c]$ values by scanning all $N_{i}[c]$ values and taking the
maximum among them.

Finally, with the $Max$ table on hand, the lower bound computation is reduced to
summing up its values. Precisely, the lower bound on the length of any
supersequence represented by a given node $x$, $L_{x}$ is equal to
$\Sigma \: Max[c]$.

One last table is used for speeding up the search process. It is used to store
the estimate values, $L_{i}$, computed for every node $i$ in the tree. This
table avoids the need to recompute these values when the search backtracks to a
certain point in the tree --- either because one of its branches has been fully
traversed or a shorter supersequence has been found.

However, once a node has been skipped or fully traversed (all of its branches
have already been examined), its $L$ value is not needed anymore. This is true
because the search will never go back to this node or any of its children again.
In fact, at any given time, the only $L$ values needed are those corresponding
to the nodes in the path to the current node. Hence, the maximum size the $L$
table can take is equal to $\mathcal{U}$ since the length of the longest path to
be traversed in the tree will be no greater than the shortest supersequence
already found.

Although the final implementation used the method described in section
\ref{sec:scs_singleletters} to compute lower bounds, it should be noted that this mechanism
can be easily extended to keep track of the number of occurrences of pairs and
triples so that the other methods described in section \ref{sec:scs_lbound} can be
used. However, it must be now clear that keeping track of pairs and triples not
only requires more memory space but also increases the time spent at each node.

\subsection{Visiting Order}

In the initial implementations of the search program, the order in
which the child nodes were traversed was the lexicographical
order, i.e. if the alphabet is $\Omega = \{A, C, G, T\}$, the first
child to be visited was the one reached with an $A$, followed by the
one reached with a $C$, and so on.

However, as mentioned earlier, the sooner a shorter supersequence is found, the
higher is the chance of pruning off more branches of the search space. With this
in mind, it was soon realized that the search could be speeded up by changing
the order in which the child nodes are visited. If the order is always the same,
say $ACGT$, the search will initially dive into the tree in a series of $A$s
until there is no more $A$s in the input sequences or the estimate of a node
exceeds the current upper bound. In fact, it takes a long time until the search
backtracks to a point where the candidate sequence is note prefixed by a long
run of $A$s.

For this reason, different orders have been tried. In the final implementation,
the order in which the children of a node are visited was set to depend on the
character that led to the current node or, in other words, on the last character
appended to the current sequence. Given a fixed permutation of the alphabet, say
$\{\tA,\tC,\tG,\tT\}$, if the last appended character was a $\tG$, the first child node
to be visited would be the one reached with a $\tT$, followed by the one reached
by an $\tA$ and so on.

In this way, the very first dive into the tree produces a candidate sequence
which is a repetition of $ACGT$. It is not difficult to see that this sequence
has a much higher chance of reaching a shorter sequence more quickly, which, as
mentioned earlier, will increase the chances of pruning off more branches of the
tree.

\subsection{Code Tuning}

The program was coded initially in Perl because of its natural suitability for
quick prototyping. It made it particularly easy to modify the program and try
different approaches, especially with regard to the lower bound estimation. When
the program reached its maturity, it was translated into C for maximum
performance.

Coding in C made it possible to fine-tune some parts of the program that used
implicit structures in Perl. For instance, it was possible to store the
sequences in a more compact fashion and to play with its memory lay-out in order
to favor cache locality.

The compact representation took advantage of the fact that the size of the
alphabet is rather small in the context this program is focused on. The number
of actual bits used for each character in memory depends on the size of the
alphabet. For instance, if the alphabet size is not larger than four (as in the
case of a DNA alphabet), only two bits are used. This means that one byte can be
used to store four characters. Similarly, if the alphabet contains only two
different characters, only one bit is used to represent a character. The
implementation determines the number of bits according to the alphabet in use (
which can be configured in the code).

This not only reduces the memory requirements but also increases the efficiency
of the memory cache since more data can fit in a cache block. For small inputs,
this compact representation gives little or no improvement at all, but as the
size of the input (number and size of sequences) gets larger, significant
improvements are observed.

Another small improvement was gained by rearranging the sequences in memory.
Initially, the sequences were stored in the obvious way, with one complete
sequence after the other. However, if we observe how these sequences are
accessed in the search�s main loop we can see that when the search goes from one
node to the other, only one character of each sequence is actually needed,
precisely those pointed by the $I$ index values. In the general case, the values
of those index values tend to have similar values most of the times. For
instance, in the beginning of the search, the first character of all sequences
will be accessed. If the sequences are organized in the obvious way, the program
will access virtually all memory blocks.

Therefore, a different memory arrangement was devised. Instead of having
complete sequences, one after the other, the program stores the first characters
of all sequences, followed by the second characters, and so on. In this way, it
is very likely that, in most of the times, very few blocks of memory will be
accessed. But, although this alternative memory lay-out may favor cache
locality, only a limited improvement could be observed in practice.

A last effort towards improving the performance of the program came from
converting the characters of the alphabet to numbers. This has to do with the
table $N$, which keeps track of the number of occurrences of letters in the
sequences. Whenever a new node is reached, the program must read the characters
in the appropriate positions in the sequences and update the corresponding
counters for those characters. In practice there is a table which is indexed by
the sequence number and the character which is being counted. For instance,
$N_{i}[c]$ gives the number of occurrences of the character $c$ in the input
sequence $p_{i}$. In C, the \texttt{char} type can be used to index an array
but, since the size of the alphabet is much smaller than the possible values of
a char variable, we must map each letter of the alphabet to a number that will
be used to index this table. And since this operation is repeated several times
as the search goes from one node to the other, even such a cheap operation (of
mapping a character to a number via another table) can, in the end, account for
a few seconds.

Hence, in order to avoid this problem, the sequences are converted --- during
the reading process --- to numbers according to the position of each character
in a fixed permutation of the alphabet. For instance, if
$\Omega = \{\tA,\tC,\tG,\tT\}$, then a sequence $\tC\tT\tT\tG\tT\tA$ is stored as $133230$.

%%%%%%%%%%%%%%%%%%%%%%%%%%%%%%%%%%%%%%%%%%%%%%%%%%%%%%%%%%%%%%%%%%%%%%%%%%%%%%%%
\section{Results}
\label{sec:scs_results}

This section will analyze how the implementation of the branch-and-bound search
algorithm described in the previous section performs in practice.

Three variables determine how much time is necessary to completely traverse the
search space with the approach outlined in the previous sections.

The length of the sequences, $m$, will ultimately affect the length of the
shortest supersequence and, therefore, how deep in the tree the program must go.
The size of the alphabet, $z$, influences the breadth of the tree. Finally, the
number of sequences, $n$, does not influence the number of explored nodes but
determines how much time is spent at each node.

It is not difficult to predict that $z$ is the most critical factor as it
increases exponentially the size of the search-space. (The number of nodes in
the $h^{th}$ level of the tree is $z^{(h - 1)}$.) As empirical results showed,
even small variations in $z$ can drastically
change the total running time. The value of $n$, on the other hand, is the less
critical one since the work done at each node is nearly proportional to $O(n)$.

Fortunately, since this work is focused on a DNA chip production setting, the
values taken by $z$ and $m$ are relatively small, although $n$ is way greater
than in any other known previous study. As described earlier,
$m$ is restricted to between 20 and 30, $z$ is equal to 4 (DNA alphabet) and $n$
is between 10,000 and 1,000,000. These values give a glimmer of hope that it
might be possible to compute an exact solution to the SCS.

Table \ref{tab:runningTimes} shows the running times of several experiments
performed on a computer equipped with a Pentium III 850 GHz processor and 256 MB
of memory.

\begin{table}
\centering
\begin{tabular}{|c|c|r|c|c|c|r|c|} \hline
\# & z & n      & m  & Heuristic & SCS & Time      & Notes \\ \hline \hline
1 & 3 & 1,000   & 10 & 28        & 27  & 5 sec.    & \\
2 & 3 & 10,000  & 10 & 29        & 28  & 13 sec.   & \\ 
3 & 4 & 10,000  & 10 & 36        & 36  & 37 min.   & \\ 
4 & 3 & 10,000  & 15 & 40        & 39  & 6.25 min. & \\
5 & 3 & 100     & 17 & 40        & 39  & 34 min.   & \dag \\
6 & 3 & 1,000   & 20 & 53        & ?   & ? hours   & \S \\ \hline \hline
\multicolumn{8}{|l|}{\dag: SCS found in less than 1 min.} \\
\multicolumn{8}{|l|}{\S: 50-char sequence found in less than 1 min.} \\ \hline \hline
\end{tabular}
\caption{\textbf{Running times with random sequences.} The first
four columns show the experiment number, the size of the alphabet (z), the
number of sequences (n) and their length (m). The fifth column shows the length
of the shorter supersequence found by the heuristics methods,
whereas the sixth column displays the length of the shortest common
supersequence. The sixth column gives the total running time.}
\label{tab:runningTimes}
\end{table}

Several interesting conclusions can be drawn from these results. Comparing the
first two experiments we can conclude that, in fact, the impact of the number of
sequences ($n$) in the running time is very small. Although the number of
sequences in the second experiment was 10 times as larger, the total time was
increased by a factor of only 2,5.

On the other hand, comparing experiments 2 and 3 we can verify that increasing
the size of the alphabet ($z$) by only 1 character can increase the time by a
factor of 170. A similar comparison between experiments 2 and 4 reveals that the
impact of increasing the length of the sequences ($m$) is also significant.

It is not difficult to see that, as the values of $z$, $n$ and $m$ approach the
values stated earlier, the time spent by the program to
search for the SCS becomes impractical. Even with an alphabet of size three and
a reduced input of only 1,000 sequences of length 20, the program was not able
to finish the search after several hours (experiment 6). In fact, from the point
where it was interrupted it was possible to conclude that it would take more
than one day to terminate.

%%%%%%%%%%%%%%%%%%%%%%%%%%%%%%%%%%%%%%%%%%%%%%%%%%%%%%%%%%%%%%%%%%%%%%%%%%%%%%%%
\section{Disscussion}
\label{sec:scs_disscussion}

In order for this program to be deemed practical, we could accept, in the
extreme case, running times  of up to one or two days, perhaps even a week. An
extreme case would be, for instance, when a newly DNA microarray chip is
designed to be produced in large scale. In this case, it would make sense to
spend a few days before starting the manufacturing of several thousand chips
since even a reduction of one step in the synthesis process could, in the end,
save a significant amount of money in the production line. And it does not seem
naive to believe that some days may be arranged for this computation during the
design phase or perhaps in parallel with other activities in the project of such
a chip.

However, judging from how an increase in the three variables $z$, $n$ and $m$,
affect the total time, we can conclude that this implementation would take
several days or maybe years to terminate with a real-world input data and,
therefore, in its current state, it cannot be applied to solve the shortest
common supersequence problem in the context of DNA microarrays.

Having said that, one good feature of this approach must be pointed out. From
the results of table \ref{tab:runningTimes} we can see that improvements of one,
two characters or even mode are usually possible for small inputs. We cannot
prove that this is also the case for real-world problem instances simply
because, at the moment, the program cannot terminate in a reasonable amount of
time. However, it is not unusual to find, even for very large inputs, an
improvement of one or two characters in the first minutes of execution.

For instance, experiment 5 shows that the SCS, which is one character shorter
than the one found by the heuristic methods, was found in the first minute of
execution. Similarly, in experiment 6, only 1 minute after a 53-character long
sequence was found with the heuristics, the program could find a supersequence
with only 50 characters. Of course, this is not always the case, but the
frequency in which this happens is not rare.

There is an intuitive explanation for this behavior. Most of the supersequences
found by the heuristic methods have a prefix which is just a repetition of a
certain permutation of the alphabet --- this characteristic is the result of how
these methods build such sequences. This prefix is then followed by suffix which
is constructed in a more or less accidental way.

As described earlier, the order in which the child nodes
are visited in the final implementation of the program also produces a sequence
whose prefix has the same characteristic, i.e. a repetition of a certain
permutation of the alphabet. When the input is large, within a couple of hours,
the program can only investigate a certain branch of the tree which corresponds
to a set of sequences with such a prefix. However, within this time, the program
is able to investigate and try all different possibilities of suffixes and,
therefore, it is usually able to shorten the supersequence. The problem is that,
again, the program takes too much time to come back to a point where it can try
different prefixes.

Finally, it should be mentioned that this approach also has the ability to prove
very quickly that no sequence of a given length exists for the input set, if
this length is very short. Say a given input set is known to have a
supersequence of length 50. If we want to know whether there exists a
supersequence of length 40, for instance, we can search in the tree with an
initial value of $\mathcal{U}$ set to 40. This will prune off all branches whose
root node has an estimate larger than 40 and, consequently, the program will
terminate much faster. Of course, if there is a sequence of length 40 (or less),
such a sequence will be found.

Other code-tuning techniques may be applied to make the current implementation
faster but clearly, in order to make this whole approach feasible, a better way
of computing the lower bounds for each node of the search tree must be devised,
one that not only finds tighter bounds --- so that more branches of the tree can
be pruned off --- but also one that can be computed quickly --- so that less
time is spent at each node.

As discussed in Chapter \ref{ch:affy}, there is evidence that Affymetrix
design their arrays by selecting probes that fit a fixed deposition sequence,
regardless of how large the probe set is. It is thus unlikely that a shorter
deposition sequence could be found for any GeneChip array. This approach
clearly restricts the sequences that can be used as probes  

Instead of fixing the probe sequences and looking for
the shortest supersequence, \citet{Tolonen2002} proposed a different approach
to reduce the length of the
deposition sequence. Their method consists of initially defining a set of
probe sequences that could be used to query
each gene of interest. Then, iteratively, a single probe or a sub-set of probes
that require the minimum number of synthesis step is selected.
