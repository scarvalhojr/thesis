%%%%%%%%%%%%%%%%%%%%%%%%%%%%%%%%%%%%%%%%%%%%%%%%%%%%%%%%%%%%%%%%%%%%%%%%%%%%%%%%
\chapter{Merging Placement and Re-embedding}
\label{ch:merge}
%%%%%%%%%%%%%%%%%%%%%%%%%%%%%%%%%%%%%%%%%%%%%%%%%%%%%%%%%%%%%%%%%%%%%%%%%%%%%%%%

The problem with the traditional ``place and re-embed'' approach is that the
arrangement of probes on the chip is decided based on embeddings that are
likely to change during the re-embedding phase. Intuitively, better results
should be obtained when the placement and embedding phases are considered
simultaneously instead of separately. However, because of the generally high
number of embeddings of each single probe, it is not easy to design algorithms
that efficiently use the additional freedom and run reasonably fast in
practice.

We describe \Greedyplus , the first placement algorithm that simultaneously
places and re-embeds the probes, and compare it with Row-Epitaxial, the best
known large-scale placement algorithm.

\section{\Greedyplus}
\label{sec:merge_greedyplus}

The goal is to design an algorithm that is similar to Row-Epitaxial, so that we
can make a better assessment of the gains resulting from merging the placement
and re-embedding phases.

\Greedyplus\ fills the spots row-by-row, from left to right, in a
greedy fashion, similarly to Row-Epitaxial. Also, for each spot $s$,
it looks at $Q$ probe candidates and chooses the one that can be
placed at $s$ with minimum cost. The difference is that we now
consider all possible embeddings of a candidate $p$ instead of only
$p$'s initial embedding. This is done by temporarily placing $p$ at
$s$ and computing its optimal embedding with respect to the
already-filled neighbors of $s$ (using OSPE from
Sec.~\ref{sec:reembed_ospe}).

Compared to Row-Epitaxial, \Greedyplus\ spends more time evaluating each probe
candidate $p$ for a spot $s$. While Row-Epitaxial takes $O(T)$ time to compute
the conflict index or the border length resulting from placing $p$ at $s$,
\Greedyplus\ requires $O(\ell T)$ time since it uses OSPE (recall that $\ell$
is the probe length and $T$ is the length of the deposition sequence). To
achieve a running time comparable to Row-Epitaxial, we must therefore consider
lower candidate numbers $Q$.

There are a few optimizations that reduce the time spent with OSPE
computations when several probe candidates are examined in succession
for the same spot. First, we note that if two probe candidates $p$ and
$p'$ share a common prefix of length $l$, the first $l + 1$ rows of
the OSPE dynamic programming matrix $D$ will be identical. In other
words, if we have calculated the minimum cost of $p$, we can speed up
the calculation of the minimum cost of $p'$ by skipping the first $l +
1$ rows of $D$.

In order to fully exploit this fact, we examine the probes in
lexicographical order so that we maximize the length of the common
prefix between two consecutive candidates. We keep a doubly-linked
list of probes and remove a probe $p$ from the list when it is placed.
For the next spot to be filled, we look at $Q$ probes in the list
around $p$'s former position, e.g., at $Q/2$ probes to the left and to
the right of $p$.

Second, the $U_t$ costs of OSPE need to be computed only once for a given spot
$s$ since $U_t$ does not depend on the probe placed at $s$. Thus, in order to
examine another candidate, we only need to recompute the $M_{i,t}$ costs.

Finally, once we know that a probe candidate $p$ can be placed at $s$
with minimum cost $C$, we can stop the OSPE computation for another
candidate $p'$ as soon as all values in a row of $D$ are greater than
or equal to $C$.


\section{Results}
\label{sec:merge_results}

\begin{table}[t]\centering
\caption{\label{tab:reptxplus}
  Normalized border length (NBL) and average conflict index (ACI) 
  of layouts produced by Row-Epitaxial and \Greedyplus\ placement (Pl.), 
  followed by Sequential re-embedding (Re-emb.)
  with thresholds $W = 0.1\%$ for border length minimization
  and $W = 0.5\%$ for conflict index minimization. $Q$ is the number of
  probe candidates considered for each spot during placement.
  Running times are given in seconds.}
\begin{tabular*}{\hsize}{lrrrr}  % formerly {@{\extracolsep{\fill}}lrrrr}
\textbf{Border length min.} & \multicolumn{2}{c}{$335 \times 335$ (E.Coli)} & \multicolumn{2}{c}{$515 \times 515$ (Maize)} \cr
\hline
\textbf{Row-Epitaxial.}\; $Q$ &    10 K &   20 K   &  10 K     &  20 K    \cr
Time (Pl. + Re-emb.)     & 629 + 9 & 1211 + 9 & 3333 + 38 & 6806 + 38\cr
NBL (Pl.)                & 34.11   & 33.81    & 33.11     & 32.87    \cr
NBL (Pl. + Re-emb.)      & 33.93   & 33.66    & 32.95     & 32.73    \cr
\hline
\textbf{\Greedyplus.}\; $Q$            & 350     &   700    &   350     &    700   \cr
Time (Pl. + Re-emb.)     & 596 + 12& 1158 + 12& 2633 + 53 & 4974 + 53\cr
NBL (Pl.)                & 33.79   & 33.22    & 33.07     & 32.38    \cr
NBL (Pl. + Re-emb.)      & 33.53   & 32.98    & 32.82     & 32.16    \cr
\hline
\cr
\textbf{Conflict index min.} & \multicolumn{2}{c}{$335 \times 335$ (E.Coli)} & \multicolumn{2}{c}{$515 \times 515$ (Maize)} \cr
\hline
\textbf{Row-Epitaxial.}\; $Q$ & 5 K        & 10 K        & 5 K         & 10 K       \cr
Time (Pl. + Re-emb.)          & 930 + 1169 & 1732 + 1167 & 4082 + 4424 & 7856 + 4415\cr
ACI (Pl.)                     & 584.92     & 544.93      & 604.04      & 574.68     \cr
ACI (Pl. + Re-emb.)           & 544.23     & 514.10      & 554.87      & 532.74     \cr
\hline
\textbf{\Greedyplus.}\; $Q$ &    200 &     300 &    200 &    300 \cr
Time (Pl. + Re-emb.)        & 522 + 788 & 685 + 788 & 2131 + 2926 & 2757 + 2930\cr
ACI (Pl.)                   & 462.52 &  450.15 & 459.38 & 446.76 \cr
ACI (Pl. + Re-emb.)         & 458.02 &  445.98 & 454.84 & 442.55 \cr
\hline
\end{tabular*}
\end{table}

We compare the results of \Greedyplus\ with Row-Epitaxial. To be fair,
since Row-Epitaxial is a traditional placement algorithm that does not
change the embeddings, we need to compare the layouts obtained by both
algorithms after a re-embedding phase. For this task we use the
Sequential algorithm (Sec.~\ref{sec:reembed_alg}) with
thresholds of $W=0.1\%$ for border length minimization and $W=0.5\%$
for conflict index minimization, so that the algorithm stops as soon
as the improvement in one iteration drops below $W$.

Table~\ref{tab:reptxplus} shows the total border length and the
average conflict index of layouts produced by both algorithms on two
chips with dimensions $335 \times 335$ and $515 \times 515$, filled
with probes randomly selected from existing GeneChip arrays (E.Coli
Genome 2.0 and Maize Genome, respectively). Probes are initially
left-most embedded into the standard 74-step Affymetrix deposition
sequence \{TGCA\}$^{18}$TG. The parameter $Q$ is chosen differently
for both algorithms so that the running time is approximately
comparable (e.g., for border length minimization, $Q=350$ for
\Greedyplus\ corresponds to $Q=10\,000$ for Row-Epitaxial). We make
the following observations.

First, increasing $Q$ linearly increases placement time, while only
marginally improving chip quality for border length minimization.

Second, re-embedding runs very quickly for border length minimization,
even on the larger chip. For conflict index minimization, the time for
the re-embedding phase exceeds the time for the placement phase for
both algorithms.

Finally, \Greedyplus\ always produces better layouts in the same amount of
time (or less) while looking at fewer probe candidates.  In particular, for
conflict index minimization on the $515\times 515$ chip with $Q=5\,000$ resp.\ 
$200$, \Greedyplus\;and Sequential improve the average conflict index by 18\%
(from 554.87 to 454.84) and need only 60\% of the time, compared to
Row-Epitaxial and Sequential.
