%%%%%%%%%%%%%%%%%%%%%%%%%%%%%%%%%%%%%%%%%%%%%%%%%%%%%%%%%%%%%%%%%%%%%%%%%%%%%%%%
\chapter*{Foreword}
\addcontentsline{toc}{chapter}{Foreword}
%%%%%%%%%%%%%%%%%%%%%%%%%%%%%%%%%%%%%%%%%%%%%%%%%%%%%%%%%%%%%%%%%%%%%%%%%%%%%%%%

Microarrays are a ubiquitous tool in molecular biology with a wide range of
applications on a whole-genome scale including high-throughput gene expression
analysis, genotyping, and resequencing. Although several different microarray
platforms exist, we focus on high-density oligonucleotide arrays, sometimes
called DNA chips. One of the advantages of higher density arrays is that they
allow the simultaneous measurement of the expression of several thousand genes
at once, possibly covering all genes of a species in a single experiment.

Oligonucleotide microarrays consist of short DNA molecules, called
\emph{probes}, affixed or synthesized at specific locations of a solid support.
Probes are built, nucleotide-by-nucleotide, by a light-directed combinatorial
chemistry. Because of the natural properties of light, the quality of a
microarray can be compromised if the physical arrangement of the probes on the
array and their synthesis schedule are not carefully designed. This thesis is
mainly concerned with the problem of designing the layout of a microarray in
such a way that the incidence of the \emph{unintended illumination problem} is
reduced. We call it the \emph{microarray layout problem} (MLP), using the term
\emph{layout} to refer to where and how the probes are synthesized on the array,
i.e., their arrangement and their \emph{embeddings}.

In the first chapter of this thesis, we briefly review the role of microarrays
in analyzing complex genetic information. We then describe the technology
currently employed in the production of high-density microarrays as well as the
problems that arise during manufacturing. In Chapter \ref{ch:mlp}, we give a
formal definition to the microarray layout problem and describe in detail two
quality measures that are used to evaluate a given layout. Finding an optimal
layout with respect to any of these two measures seems unlikely, even for very
small arrays. As we shall see in Chapter \ref{ch:qap}, the MLP can be modeled as
a quadratic assignment problem (QAP), a classical combinatorial optimization
problem that is notoriously hard to solve in practice, giveing further
indication that the MLP is, in fact, a hard problem. In practice, the layout
problem is usually approached in several ``phases'' with a range of heuristic
algorithms.

The \emph{placement} phase is the subject of Chapter \ref{ch:placement}.
Traditionally, this phase consists of fixing an embedding for all probes and
finding an arrangement minimizing a given cost function. We describe several
known placement algorithms with an emphasis on methods that can be used to
design large arrays. A new algorithm, called Greedy, is also presented. One of
the reason why we show the relation between the MLP and the QAP is that we can
now use QAP techniques as placement algorithms. This is interesting because
there is a rich literature on methods for solving the QAP. In Chapter
\ref{ch:qap}, we also show the results of using one QAP heuristic to design
small artificial chips, and discuss how this approach can be applied to larger
microarrays.

Chapter \ref{ch:reembed} focuses on the \emph{re-embedding} phase that usually
follows the placement. In this phase, one attempts to further improve the layout
by finding a different embedding of the probes without changing their location
on the chip. Again, we review all known re-embedding algorithms, describring the
most successful ones in detail. We also introduce a new algorithm, called
Priority re-embedding.

In the last decade, commercial microarrays have grown from a few thousands to
more than a million probe sequences on a single chip. Many placement algorithms
are unable to deal with such large arrays because of their non-linear time and
space complexities. For this reason, the layout problem is sometimes broken into
smaller sub-problem by a \emph{partitioning} algorithm. This is the focus of
Chapter \ref{ch:part}, where we present an extensive evaluation of existing
algorithms and show how the partitioning phase can improve solution quality and
reduce running time.

In Chapter \ref{ch:merge}, we discuss the disadvantages of the tradional ``place
and re-embed'' approach to the layout problem. We then propose a new algorithm,
called \Greedyplus, that for the first merges the placement and re-embedding
phases into a single one. Our results show that \Greedyplus\ indeed outperforms
all known placement algorithms.

In Chapter \ref{ch:affy}, we present a pioneering analysis and evaluation of the
layout of several GeneChip\textR\ arrays, considered the industry standard in
terms of high-density oligonucleotide microarrays. Some design decisions that
might affect the quality of these arrays are described in detail. We then use
some of the algorithms presented in earlier chapters to propose alternative
layouts for two of the latest generation of GeneChip arrays, showing how the
risk of the unintended illumination problem can be reduced.

Another problem related to the production of microarrays is to find the shortest
synthesis schedule for a given set of probes, which we refer to as the
\emph{shortest deposition sequence problem} (SDSP). The SDSP is an instance of
the shortest common supersequence problem (SCSP), a classical problem in
computer science that is known to be NP-complete even under various
restrictions. Several existing heuristics are able to find good approximate
solutions for the SCSP, but, in Chapter \ref{ch:scs}, we investigate the
feasibility of finding \emph{the shortest} deposition sequence for currently
available oligonucleotide microarrays. Chapter \ref{ch:discuss} concludes this
thesis with a short discussion about the presented results.

\paragraph{Publications.}
Parts of this thesis have been published in advance. The conflict index model
for evaluating a microarray layout (Chapter \ref{ch:mlp}) and the Pivot
Partitioning algorithm (Section \ref{sec:part_pp}) were first presented at the
Workshop on Algorithms in Bioinformatics (WABI), in Z\"urich
\citep{Carvalho2006}. The conflict index model was also presented, together with
the QAP formulation of the microarray layout problem (Chapter \ref{ch:qap}), at
the German Conference on Bioinformatics (GCB) in T\"ubingen
\citep{Carvalho2006a}. The work on the shortest common supersequence (Chapter
\ref{ch:scs}) was first published as a technical report at the Faculty of
Technology of Bielefeld University \citep{Carvalho2005}. Finally, a book chapter
containing a more accessible description of the microarray layout problem and of
several algorithms presented here, including the previously unpublished
\Greedyplus, 1-Dimensional and 2-Dimensional Partitioning, is expected to appear
in late 2007 \citep{Carvalho2007}.

This thesis also contains previously unpublished material, namely:
\begin{itemize}
\item the Greedy placement algorithm (Section \ref{sec:placement_greedy}) that
      outperforms previous algorithms in terms of conflict index minimization;
\item the Priority re-embedding algorithm (Section \ref{sec:reembed_priority})
      that achieves marginal improvements compared to the best known algorithms;
\item an analysis of the layout of several commercially available GeneChip
      arrays with respect to the defined evaluation criteria (Chapter
      \ref{ch:affy}).
\end{itemize}

\paragraph{Acknowledgments.}
This work was carried out while I was a member of the Junior research group
(recently-renamed) Computational Methods for Emerging Technologies (COMET), which
is part of the AG Genominformatik led by Prof.~Jens Stoye. I thank all present
and former colleagues as well as students of the International NRW Graduate
School in Bioinformatics and Genome Research and the Graduiertenkolleg
Bioinformatik, of which I am also a member, for the nice research atmosphere I
found in Bielefeld, and for an enjoyable time I had in the last three years.

Special thanks go to Dr.~Sven Rahmann for suggesting the topic and for the
opportunity to work under his supervision. This work owes much to his expertise.
Whenever I write ``we'' in this thesis, I mean ``Sven and I''. On several
occasions, the support of the Bioinformatics Resource facility (BRF) at the
CeBiTec (Center for BioTechnology) was crucial to the success of this work,
and I cannot thank them enough for their help. Epameinondas Fritzilas,
Francisco Pereira Lobo, Ferdinando Cicalese, Jos\'e Augusto Amgarten Quitzau,
and Klaus-Bernd Sch\"urmann read early drafts of several chapters of this thesis
and helped to improve them in many ways. I would also like to thank Dr.~Peter
Hahn and Chris MacPhee for working on several QAP instances of
Chapter~\ref{ch:qap} and for helpful discussions.

\vspace*{6ex}
S\'ergio A. de Carvalho Jr.\hfill Bielefeld, \handindate
