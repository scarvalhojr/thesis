%%%%%%%%%%%%%%%%%%%%%%%%%%%%%%%%%%%%%%%%%%%%%%%%%%%%%%%%%%%%%%%%%%%%%%%%%%%%%%%%
\chapter*{Foreword}
\addcontentsline{toc}{chapter}{Foreword}
%%%%%%%%%%%%%%%%%%%%%%%%%%%%%%%%%%%%%%%%%%%%%%%%%%%%%%%%%%%%%%%%%%%%%%%%%%%%%%%%

Microarrays are a ubiquitous tool in molecular biology with a wide range of
applications on a whole-genome scale including high-throughput gene expression
analysis, genotyping, and resequencing.

Several different microarray platforms exist, but this thesis focuses on
high-density oligonucleotide arrays. The advantage of higher densities arrays
is that they allow, for instance, the simultaneous measurement of the expression
of several thousand genes at once.

High-density microarrays are usually produced by light-directed combinatorial
chemistry that builds the probe sequences base-by-base. Because of the natural
properties of light, the quality of a microarray can be improved by carefully
designing the physical arrangement, or \emph{layout}, of its probes.

In this thesis, we review models for evaluating the layout of oligonucleotide
microarrays and survey algorithmic approaches that can be used in their design.

%% What follows is from Sven's thesis: use it as a template

DNA chips allow to quickly obtain and compare gene expression profiles
of different cell or tissue samples. As one of many applications, one
hopes to better understand various types and subtypes of cancer and to
improve cancer therapy by characterizing the differences between
the expression profiles of healthy cells and tumor cells.

The first chapter of this thesis offers an overview of existing
microarray technologies and contrasts them with other techniques for
gene expression analysis. High-density oligonucleotide arrays are
described in detail.

Gene expression analysis with DNA chips is a high-throughput technique
that produces massive amounts of data; however, this technique is not
error-free. Each step of an experiment must be performed carefully. In
particular, the DNA chip must be carefully designed and manufactured.
This thesis proposes and describes solutions for some of the
algorithmic problems that arise in this phase. These problems are
formulated in detail in the last section of Chapter x.

My research started with the general question of how to find
oligonucleotide probes for highly homologous transcripts when it
cannot be guaranteed that a sufficiently large set of clearly
transcript-specific (or \emph{unique}) probes can be found. An
interesting future large-scale application could be the individual
expression measurement of all splice variations of all genes in the
human genome, for example. The basic idea was to find several
\emph{non-unique} probes such that different probes hybridize to
different combinations of targets, with the hope that the measured
signal can then be decoded into the individual expression levels. Two
question then follow immediately: How does the decoding work, and how
can the probes be designed in a way that makes the decoding as simple
as possible?

In early 2001, custom probe design was in its infancy. The design of
the large commercial chips, such as the Affymetrix GeneChip\textR, was
a well guarded secret of the respective companies. Several other
research groups were also beginning to work on probe selection, and
their results, made the
problem more popular.  I soon realized that, before actually working
on non-unique probes, more fundamental questions should be addressed.
A high-performance large-scale probe selection system for unique
probes would be very useful but did not exist. Such a system could
then be used as a basis for non-unique probe design. These
considerations are reflected in this thesis.

Once a set of probes is chosen for a chip design, the chip must be
produced. With several technologies, such as the Affymetrix
GeneChip\textR\ arrays and febit's {\sffamily geniom}\textR\ 
{\sffamily one} system, the probes are synthesized in situ on the chip
with a combination of photolithography and combinatorial chemistry.
We consider the problem of optimizing the nucleotide deposition
sequence to lower production costs and to decrease the overall error
rate during synthesis.

A concluding discussion about the results of this thesis and an
outlook into the future can be found in Chapter~\ref{ch:discussion}.

\paragraph{Publications.}
Parts of this thesis have been published in advance.

This thesis also contains previously unpublished material, namely
\begin{itemize}
\item new placement algorithm that simultaneously places and re-embeds
  the probes
\end{itemize}

\paragraph{Acknowledgments.}
This work was carried out while I was at the...
It has been a very enjoyable experience, thanks to all present and
former colleagues and students in the Bioinformatics program.

I would especially like to thank Sven Rahmann for suggesting the
topic, providing initial ideas, and for the opportunity to write this
thesis under his guidance.

The support of the whole computer service group at the CeBiTec was
crucial to the success of this work, and I cannot thank them enough
for their help.

I enjoyed fruitful discussions with many people while this thesis was
underway; in particular, I am grateful to ...
for their valuable hints and comments. 

... read early drafts of several chapters of this thesis and
helped to improve it in many ways through their comments.

I would like to thank Dr. Peter Hahn and Chris MacPhee who worked on several QAP instances
derived from artificial microarray chips for their efforts.

\ignore{
Last but not least, I thank Karla Aleluia de Carvalho for her love and
patience while I was writing this thesis. Heartfelt thanks also go to
my parents and the rest of the family.}

\vspace*{6ex}
S\'ergio A. de Carvalho Jr.\hfill Bielefeld, \handindate
