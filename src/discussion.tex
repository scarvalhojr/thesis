\chapter{Discussion}
\label{ch:discussion}

This thesis makes several contributions to the field of....

We introduce the \emph{conflict index}...

In Chapter X we present a algorithm...

We are not aware of any previous work that combines placement and
re-embedding, even our results showed that such an approach has
obvious benefits...

We have surveyed algorithms for the microarray layout problem (MLP), divided
into placement, (re-)embedding, and partitioning algorithms.  Because of the
super-exponential number of possible layouts and the relation to the quadratic
assignment problem (QAP), we cannot expect to find optimal solutions. Indeed,
the algorithms we present are heuristics with an emphasis on good scalability
and ideally a user-controllable trade-off between running time and solution
quality, albeit without any known provable guarantees.

Among the presented approaches, two recent ones (Pivot Partitioning and
\Greedyplus) indicate that the traditional ``place first and then re-embed''
approach can be improved upon by merging the partitioning/placement and
(re-)embedding phases. Ongoing work will show the full potential of such
combined approaches.

As a suggestion for further work, we note the needed for improving
the selection of probe candidates considered for filling each spot. For
example, instead using a sorted list of probes, one could use a TSP tour
like the early algorithms described in Sec.~\ref{sec:placement_early}. However,
it is not clear if the more time-consuming TSP approach will pay off
(instead, we could use this extra time to look at more candidates).

An alternative that sounds interesting would be to build some kind of
``clustering'' of the probes, perhaps based on a graph or a tree, in such a
way that we can find similar probes more easily and spend time on candidates
that are more likely to produce less conflicts.

Question: why PM and MM probes are placed together and aligned except for the
middle base?