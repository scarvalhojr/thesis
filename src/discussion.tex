%%%%%%%%%%%%%%%%%%%%%%%%%%%%%%%%%%%%%%%%%%%%%%%%%%%%%%%%%%%%%%%%%%%%%%%%%%%%%%%%
\chapter{Discussion}
\label{ch:discuss}

We have focused on two computational problems related to the production of
oligonucleotide microarrays: the microarray layout problem (MLP) and the
shortest deposition sequence problem (SDSP). With respect to the former, this
thesis constitutes a detailed study of strategies and algorithmic approaches
that can be used to design the layout of high-density microarrays. Because of
the super-exponential number of possible layouts and the relation to the
quadratic assignment problem (QAP), we cannot expect to find optimal solutions.
Indeed, the algorithms we presented are heuristics with an emphasis on good
scalability and, ideally, a user-controllable trade-off between running time and
solution quality, albeit without any known provable guarantees. We have
concentrated our work on algorithms that can handle, in reasonable time,
relatively large chips with the 25-mer probes typically found on GeneChip
arrays, presenting an extensive range of empirical results on the best known
methods. We hope that this work will help improving the quality of the next
generation of microarrays. In summary, we have made the following contributions.

\paragraph{Extended model for microarray layout evaluation.} In Chapter
\ref{ch:mlp} we gave a formal definition to the microarray layout problem and
introduced the conflict index model for evaluating a microarray layout and
estimating the risk of unintended illumination. This model extends the border
length definition of \citet{Hannenhalli2002} by taking into account the position
inside the probe where the conflict occurs and the distance between the spots.

Although adjusting this model to a particular fabrication technology is beyond
the scope of this thesis, all algorithms discussed in later chapters make no
assumption about the range of values returned by the weighting functions used in
our definition of conflict index. Consequently, our empirical results should be
reproducible using differents constants or even similarly-defined functions.

\paragraph{QAP formulation of MLP.} In Chapter \ref{ch:qap} we showed that the
microarray layout problem can be formulated as a quadratic assigment problem
(QAP). We then showed how a microarray can be designed using QAP heuristics, and
reported experimental results using a QAP algorithm, known as GRASP, to design
the layout of small artificial microarrays. Although GRASP was able to produce
good layouts, there was clearly a problem of running time, and we do not expect
any QAP algorithm to outperform the best known placement algorithms.
Nevertheless, our formulation is of interest as there is a rich literature on
QAP and numerous methods that can now be applied for the MLP. As a suggestion
for further work, we discussed how an existing layout could be improved using
our QAP approach, iteratively.

\paragraph{Algorithms.} After describing all known placement algorithms in
detail, we introduced a new algorithm, called Greedy (Section
\ref{sec:placement_greedy}), in Chapter \ref{ch:placement}. In terms of border
length minimization, Greedy achieved results comparable to Row-Epitaxial
\citep{Kahng2003}, the previously best known placement algorithm, although
Greedy was slower in our results. In terms of conflict index minimization,
however, Greedy clearly outperformed Row-Epitaxial.

Chapter \ref{ch:reembed} was devoted to the re-embedding phase that usually
follows the placement in an attempt to further reduce conflicts. After
describing all known algorithms of this kind, we introduced a new algorithm,
called Priority re-embedding. In our results, Priority achieved marginal
improvements compared to Sequential, the best re-embedding algorithm to our
knowledge. Unfortunately, the extra complexity and slower performance of
Priority make it hard to justify its use. In fact, we view these results as a
further indication that there is little room for improvements on the
re-embedding phase.

In Chapter \ref{ch:part}, we first described 1-Dimensional and 2-Dimensional
Partitioning \citep{Carvalho2007}. We demonstrated how these two algorithms can
be used to generate a few masks with extremely low levels of conflicts, which
can be especially helpful in case of conflict index minimization. We also
described two partitioning algorithms, Centroid-based Quadrisection
\citep{Kahng2003a} and Pivot Partitioning \citep{Carvalho2006}, that offer a
more uniform optimization over all synthesis steps. Earlier results on chips
with relatively long deposition sequences suggested that Pivot Partitioning is
better than Centroid-based Quadrisection, and that these algorithms improve
solution quality and reduce running times.

Our new results on chips with the shorter deposition sequence used by
Affymetrix, however, showed that the restriction in number of candidates per
probe during placement of the last spots of a region (when algorithms such as
Row-Epitaxial and Greedy are used for the placement) often impacts the solution
quality more significantly than the gains due to grouping similar probes
together. As a result, Pivot Partitioning improved solution quality only in
terms of conflict index, although it often reduced running time. Nevertheless,
we believe that there is still room for improvements on partitioning algorithms.

Our new approach to the layout problem that merges the placement and
re-embedding phases was discussed in Chapter \ref{ch:merge}, where we presented
\Greedyplus\ \citep{Carvalho2007}. Our results showed that \Greedyplus\
outperforms previous algorithms based on the traditional approach, such as
Greedy and Row-Epitaxial, in terms of border length as well as conflict index
minimization. Although Greedy might produce better results on large chips if
time is restricted, we believe that \Greedyplus\ has a greater potential for
producing the best layouts in both quality measures because it needs to examine
fewer probe candidates to achive similar results. Among all presented
algorithms, \Greedyplus\ and Pivot Partitioning and indicate that the
traditional ``place first and then re-embed'' approach can be improved upon by
merging the partitioning/placement and (re-)embedding phases.

As a suggestion for further work on placement algorithms, we note the
possibility of improving the order in which probe candidates are considered for
filling each spot by algorithms such as Row-Epitaxial, Greedy, and \Greedyplus.
Sorting the probes lexicographically tends to improve the first synthesis steps
more than the others. One possibility is to use the TSP-based approach described
in Section~\ref{sec:placement_threading}. However, it is unlikely that the
time-consuming TSP computation will pay off, especially for large chips ---
instead, we could use this extra time to look at more probe candidates. As
discussed in the end of Chapter \ref{ch:merge}, sorting the probes with an
emphasis on the middle bases is likely to improve the layouts in terms of
conflict index. For \Greedyplus, however, it remains to be seen whether a
different implementation of OSPE can be used in combination with such an
ordering without incurring in increased running times.

\paragraph{Analysis of Affymetrix microarrays.} In Chapter \ref{ch:affy} we used
the border length and conflict index quality measures to make, for the first time, an
evaluation of the layout of several GeneChip arrays. Our analysis
revealed that the design approach used by Affymetrix
evolved since the first generation of chips, probably as a result of attempting
to reduce border conflicts. We showed that the current approach of placing
perfect match (PM) and mismatch (MM) probes on adjacent spots reduces
border conflicts, but it also results in a concentration of
conflicts on the synthesis steps where an error is more likely to damage the
probes. This fact could add to the argument that the PM/MM pairing used by
Affymetrix should be dropped altogether, as some researchers have recently
proposed \citep{Lauren2003}. Although the PM probe is expected to have a higher
affinity for the specific target than the MM probe, it has been reported that
sometimes the signals from the mismatch spots are stronger than the perfect
match \citep{Naef2003}. The reliability of the PM/MM approach to account for
nonspecific hybridizations has not yet been estabilished by published
experiments, and some researches claim that comparable or better analysis are
possible without the MM signals \citep{Irizarry2003}. In fact, there is a wide
range of alternative methods for analyzing the gene expression experiments
obtained from Affymetrix chips \citep{Irizarry2006,Millenaar2006}.

Since the position of the probe on the chip bears no relation with its function,
we proposed different layouts for two of the latest GeneChip arrays, where the
PM and MM probes were allowed to occupy non-adjacent spots. Our results showed
that the Affymetrix layouts can be significantly improved, especially in terms
of conflict index. Even in terms of border length, we managed to produce layouts
with as much as $8.10\%$ less border conflicts using the algorithms presented in
earlier chapters.

\paragraph{Shortest common supersequence.}
In Chapter \ref{ch:scs}, we studied the shortest deposition sequence problem as
an instance of the shortest common supersequence problem (SCSP). Althoug several
heuristic algorithms exist for the SCSP, our goal was to determine the
feasibility of finding \emph{the shortest} deposition sequence for a given set
of probes. We employed a branch-and-bound algorithm, the only approach that
seems feasible for our setting. Our results indicate that the problem remains
intractable for a typical high-density microarray. This, however, does not seem
to be a major problem for microarray production because, commonly, a deposition
sequence is fixed even before the probe sequences are selected.

%%%%%%%%%%%%%%%%%%%%%%%%%%%%%%%%%%%%%%%%%%%%%%%%%%%%%%%%%%%%%%%%%%%%%%%%%%%%%%%%
\section{Outlook}
\label{sec:discuss_outlook}

Today, Affymetrix produces up to $1\,164\times 1\,164$ arrays in large scale,
and we have showed that good layouts for arrays of this size can be designed in
a few hours. When the best results are required, one or two days are enough,
with reasonable computing power. We expect to see larger microarrays being
produced in the near future as there is an increasing need for widening the
range of genes that can be monitored in a single experiment. Still, we belive
that this should cause no major problems in terms of layout design, for two
reasons. First, because a continuous increase in computing power should also be
expected. Second, because it is possible to control the running time of the best
algorithms presented here (Greedy and \Greedyplus), so they can be configured to
compute the best layout in the available time. 

For commercial microarrays, we believe that, even if an algorithm takes a week
to complete, it is time well spent given that they are likely to be produced in
large quantities and that the layout needs to be designed only once. This is
specially true if we consider that a week is a relatively short time compared to
the the time required for the entire design process of an off-the-shelf
microarray chip.

The fact that it is possible to control the running time of the best algorithms
is also good news for custom microarray production, because, in this case, only
a few units are usually produced, and there is an obvious need to design them as
quickly as possible. Custom chips produced today are still relatively small when
compared to chips produced in large scale. This could change as
technologies, such as the self-contained {\sffamily geniom} platform of febit
biotech GmbH, become increasingly more mature and affordable.
